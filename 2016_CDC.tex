%%%%%%%%%%%%%%%%%%%%%%%%%%%%%%%%%%%%%%%%%%%%%%%%%%%%%%%%%%%%%%%%%%%%%%%%%%%%%%%%
%
%        1         2         3         4         5         6         7         8

%\documentclass[letterpaper, 10 pt, conference]{ieeeconf}  % Comment this line out
                                                          % if you need a4paper
%\documentclass[a4paper, 10pt, conference]{ieeeconf}      % Use this line for a4
%\documentclass[10pt, conference]{ieeeconf}                                                    % paper

\documentclass[letterpaper, 10 pt, conference]{ieeeconf}

\IEEEoverridecommandlockouts                              % This command is only
                                                          % needed if you want to
                                                          % use the \thanks command
\overrideIEEEmargins
% See the \addtolength command later in the file to balance the column lengths
% on the last page of the document




\usepackage{epsfig} % for postscript graphics files
\usepackage{amsmath} % assumes amsmath package installed
\usepackage{amssymb}  % assumes amsmath package installed
\usepackage{psfrag}
\usepackage{color}
\usepackage[usenames,dvipsnames,table]{xcolor}
\usepackage{units}
\usepackage[top=54pt, left=54pt, right=54pt, bottom=54pt, paper=letterpaper]{geometry}
\usepackage{paralist}
\usepackage[table]{xcolor}
\usepackage[noadjust]{cite}
\usepackage{epstopdf}
\usepackage{booktabs}
\usepackage{tikz}
\usetikzlibrary{arrows,positioning,patterns,decorations.pathreplacing}
\usepackage{tikz-3dplot}
\usepackage[labelformat=empty]{subfig}

\newcommand{\subparagraph}{}

%\usepackage{cite}
\usepackage{titlesec}
%

\usepackage{accents}
\newlength{\dhatheight}
\newcommand{\doublehat}[1]{%
    \settoheight{\dhatheight}{\ensuremath{\hat{#1}}}%
    \addtolength{\dhatheight}{-0.25ex}%
    \hat{\vphantom{\rule{1pt}{\dhatheight}}%
    \smash{\hat{#1}}}}

%\titlespacing{\section}{3em}{2pt}{2pt}
%\titlespacing{\subsection}{1em}{1pt}{1pt}
%\titlespacing{\subsubsection}{1em}{1pt}{1pt}

%\abovedisplayshortskip=1pt
%\belowdisplayshortskip=1pt
%\abovedisplayskip=1pt
%\belowdisplayskip=1pt





%% own commands and environments
 %--------------------------------------
 
%% own macros
\newtheorem{thm}{Theorem}
\newtheorem{cor}[thm]{Corollary}
\newtheorem{lem}[thm]{Lemma}
\newtheorem{assum}{Assumption}
\newtheorem{prop}[thm]{Proposition}
\newtheorem{defn}{Definition}
\newtheorem{exmp}{Example}
\newtheorem{rem}{Remark}
\newtheorem{alg}{Algorithm}

\newcommand{\R}{\mathbb{R}} 
\newcommand{\N}{\mathbb{N}} 
\newcommand{\C}{\mathbb{C}}

\newcommand{\A}{\mathcal{A}}
\newcommand{\X}{\mathcal{X}}
\newcommand{\B}{\mathcal{B}}
\newcommand{\D}{\mathcal{D}}
\newcommand{\Cc}{\mathcal{C}}
\newcommand{\Pp}{\mathcal{P}} 
\newcommand{\Ll}{\mathcal{L}} 
\newcommand{\Rr}{\mathcal{R}} 
\newcommand{\Ss}{\mathcal{S}}
\newcommand{\W}{\mathcal{W}} 
\newcommand{\T}{\mathcal{T}}
\newcommand{\I}{\mathcal{I}}  
\newcommand{\V}{\mathcal{V}} 
\newcommand{\U}{\mathcal{U}} 
\newcommand{\Q}{\mathcal{Q}} 
\newcommand{\Y}{\mathcal{Y}}
\newcommand{\Z}{\mathcal{Z}}
\newcommand{\M}{\mathcal{M}}

\newcommand{\conv}{\mathrm{conv}}

\newcommand{\lb}[1]{\underline{#1}} 
\newcommand{\ub}[1]{\overline{#1}} 
\newcommand{\blind}[1]{\textcolor{white}{#1}}

\title{\bf Parametric robust positively  invariant sets  \\
for linear systems with scaled disturbances}

%\title{\bf Similarities and disparities among \\
%nonlinear MPC schemes with guaranteed stability}


%\author{ \parbox{3 in}{\centering Huibert Kwakernaak*
%         \thanks{*Use the $\backslash$thanks command to put information here}\\
%         Faculty of Electrical Engineering, Mathematics and Computer Science\\
%         University of Twente\\
%         7500 AE Enschede, The Netherlands\\
%         {\tt\small h.kwakernaak@autsubmit.com}}
%         \hspace*{ 0.5 in}
%         \parbox{3 in}{ \centering Pradeep Misra**
%         \thanks{**The footnote marks may be inserted manually}\\
%        Department of Electrical Engineering \\
%         Wright State University\\
%         Dayton, OH 45435, USA\\
%         {\tt\small pmisra@cs.wright.edu}}
%}

\author{Moritz Schulze Darup$^{\dagger}$, Rainer Manuel Schaich$^{\dagger}$, and Mark Cannon% <-this % stops a space
\thanks{*This research was partially funded by the German Research Foundation (DFG) under the grants SCHU 2094/1-1 and SCHU 2094/2-1.}%
% <-this % stops a space
\thanks{$^{\dagger}$ These authors contributed equally to this work.}% <-this % stops a space
\thanks{M. Schulze Darup, R. M. Schaich, and M. Cannon are with 
the Control Group, Department of Engineering Science,
        University of Oxford, Parks Road, Oxford OX1 3PJ, UK.
        E-mail: {\tt mark.cannon@eng.ox.ac.uk}.}%
}


\begin{document}
\maketitle
\thispagestyle{empty}
\pagestyle{empty}

%%%%%%%%%%%%%%%%%%%%%%%%%%%%%%%%%%%%%%%%%%%%%%%%%%%%%%%%%%%%%%%%%%%%%%%%%%%%%%%%
\begin{abstract}
Abstract
\end{abstract}

%%%%%%%%%%%%%%%%%%%%%%%%%%%%%%%%%%%%%%%%%%%%%%%%%%%%%%%%%%%%%%%%%%%%%%%%%%%%%%%%
\section{Introduction}


Robust positively invariant (RPI) sets are important for performance analysis and synthesis of controllers for uncertain systems (see, e.g., \cite[Sects.~6.4 and 6.5]{Blanchini1999}).
In particular, RPI sets can be used to design robust model predictive control (MPC) schemes with guaranteed stability (see, e.g., \cite{Kouvaritakis2015,Lee1999,Mayne2006,Mayne2005}). In this paper, we address RPI sets for linear disturbed systems of the form
\begin{equation}
\label{eq:system}
x(k+1) = A \, x(k) + w(k)
\end{equation}
 with state and disturbance constraints
\begin{equation}
\label{eq:constraints}
x(k) \in \X  \quad \text{and} \quad w(k) \in \W^\alpha \quad \text{for every} \quad k \in \N.
\end{equation}
The set $\W^\alpha := \alpha \, \W^\ast$ denotes a scaled version  of a nominal disturbance set $\W^\ast$ for some scalar $\alpha>0$.


Extension of \cite{Schaich2015} in two directions. First, here we consider general C-sets instead of polytopes as in \cite{Schaich2015}. Second, we avoid the occurrence of empty sets by construction.


In this paper, the sets $\X$ and $\W^\ast$ are assumed to be full-dimensional (see Assum.~\ref{assum:AXW} further below). While this is restrictive, the same setup was studied in \cite{Rakovic2005} % \cite{Rakovic2005,Schaich2015}.

The paper is organized as follow. We state notation in the remainder of this section.

\subsection{Notation}

We denote positive real and natural numbers by $\R_+$ and $\N_+$, respectively. The notion $\N_{[i,k]}$ refers to $\N_{[i,k]}:=\{ j \in \N \,|\, i \leq j \leq k \}$.
Now, let $p \in \N_+$ and consider two bounded and convex sets $\U , \V \subset \R^p$ containing the origin. We frequently use the following manipulations of sets.
 The scaling of a set by some factor $\beta \in \R_+$  is defined as $\beta \,\U := \{ \beta u \in \R^p \,|\, u \in \U\} $.
Moreover, for $q \in \N_+$ and some matrices $C \in \R^{p\times p}$ and $D \in \R^{q \times p}$, we define $C^{-1} \, \U:= \{ \xi \in \R^p \,|\, C\,\xi \in \U \}$ and  $D\,\U := \{ D u \in \R^q \,|\, u \in \U \} $. Note that $C^{-1} \, \U$ is well defined even if $C$ is not invertible. The operations
\begin{align*}
\U \oplus \V &:= \{ u+v \in \R^p \,|\,  u \in \U, v \in \V\} \qquad \text{and} \\
\U \ominus \V &:= \{ \xi \in \R^p \,|\, \forall v \in \V: \xi+v \in \U\},
\end{align*}
describe the Minkowski addition and the Pontryagin difference, respectively. Finally, the support function of the set $\U$ and a row vector $c \in \R^{1 \times p}$ is defined as
$$
h_{\U}(c) := \sup_{u \in \U} c\, u. 
$$
When used with a matrix argument, $h_{\U}(D)$ is understood as
$$
h_{\U}(D) := \begin{pmatrix}
h_{\U}(e_1^T D) & \dots & h_{\U}(e_q^T D)
\end{pmatrix}^T.
$$
Eventually, we use the following shorthand notations.
A convex and compact set containing the origin as an interior point is called a C-set. 
The matrix $I_p$ refers to the identity matrix in $\R^{p \times p}$. 

\section{Preliminaries}


\begin{defn}
\label{def:RPI}
Let $\X, \W^\ast \subseteq \R^n$ and $\alpha \in \R_+$.
The set $\Pp$ is called robust positively invariant (RPI) for system~\eqref{eq:system} with constraints~\eqref{eq:constraints}, if (i) $\Pp \subseteq \X$ and (ii) $A\, x+w\in \Pp$ for every $x \in \Pp$ and every $w \in \W^\alpha$.
\end{defn}

Note that both the union and the intersection of two RPI sets result in another RPI set. Further note that $\Pp=\emptyset$ is a RPI set according to Def.~\ref{def:RPI}. Based on these observations, the following definitions of the maximal and minimal RPI set are reasonable. 

\begin{defn}
\label{def:MRPI}
Let $\X, \W^\ast \subseteq \R^n$ and $\alpha \in \R_+$. The union of all RPI sets for~\eqref{eq:system} with~\eqref{eq:constraints} is called the maximal robust positively invariant (MRPI) set for~\eqref{eq:system} with~\eqref{eq:constraints} and denoted by $\Pp_{\max}^\alpha$.
\end{defn}

\begin{defn}
\label{def:mRPI}
Let $\X, \W^\ast \subseteq \R^n$ and $\alpha \in \R_+$. The intersection of all nonempty RPI sets and the MRPI set for~\eqref{eq:system} with~\eqref{eq:constraints} is called the minimal robust positively invariant (mRPI) set for~\eqref{eq:system} with~\eqref{eq:constraints} and denoted by $\Pp_{\min}^\alpha$.
\end{defn}

Two important statements can be inferred  from Defs.~\ref{def:MRPI} and \ref{def:mRPI}. First, $\Pp_{\min}^\alpha$ is empty if and only if $\Pp_{\max}^\alpha$ is empty. Second, $\Pp_{\min}^\alpha \subseteq \Pp_{\max}^\alpha$.

\begin{assum}
\label{assum:AXW}
The eigenvalues $\lambda \in \C$ of the matrix $A \in \R^{n \times n}$ are strictly stable (i.e., $| \lambda|<1$).
The sets $\X$ and $\W^\ast$ are C-sets in $\R^n$.
\end{assum}


\begin{equation}
\label{eq:sequenceRk}
\Rr_{k+1}^\alpha := A\, \Rr_k^\alpha \oplus \W^\alpha \qquad \text{with} \qquad  \Rr_{0}^\alpha:=\{0\}.
\end{equation}

We obviously have $\Rr_k^\alpha \subseteq \Rr_{k+1}^\alpha$.


\begin{equation}
\label{eq:sequenceSk}
\Ss_{k+1}^\alpha := A^{-1} (\Ss_k^\alpha \ominus \W^\alpha) \cap \X \quad \text{with} \quad  \Ss_{0}^\alpha:=\X,
\end{equation}

We obviously have $\Ss_k^\alpha \supseteq \Ss_{k+1}^\alpha$.

\begin{lem}
\label{lem:xInSkAlpha}
Let Assum.~\ref{assum:AXW} be satisfied, let  $k \in \N_+$ and $\alpha \in \R_+$, and let $x_0 \in \X$. Then $x_0 \in \Ss_k^\alpha$ if and only if
$$
A^j x_0 + \sum_{i=0}^{j-1} A^{j-1-i} w(i)  \in \X
$$
for every $j \in \N_{[1,k]}$ and every disturbance sequence $w(0),\dots, w(k-1) \in \W^{\alpha}$.
\end{lem}

\begin{equation}
\label{eq:RLimit}
\Rr_\infty^\alpha:=\lim_{k \rightarrow \infty} \Rr_k^\alpha =  \bigcup_{k=0}^\infty \Rr_k^\alpha
\end{equation}

\begin{equation}
\label{eq:SLimit}
\Ss_\infty^\alpha:=\lim_{k \rightarrow \infty} \Ss_k^\alpha  = \bigcap_{k=0}^\infty \Ss_k^\alpha 
\end{equation}

\begin{lem}
\label{lem:PmaxPmin}
Let Assum.~\ref{assum:AXW} be satisfied. Then, $\Pp_{\max}^\alpha=\Ss_\infty^\alpha$ and 
\begin{equation}
\label{eq:Pmin}
\Pp_{\min}^\alpha=\left\{ \begin{array}{ll}
\Rr_{\infty}^\alpha & \text{if} \quad \Rr_{\infty}^\alpha\subseteq \X, \\
\emptyset & \text{otherwise}.
\end{array} \right.
\end{equation}
\end{lem}

From \cite[Eqs.~(5.1) and (5.2)]{Kolmanovsky1998}, we infer
\begin{equation}
\label{eq:relationSkRk}
\Ss_{k}^\alpha = \bigcap_{j=0}^k (A^{j})^{-1} (\X \ominus \Rr_{j}^\alpha).
\end{equation}
Since the statement is not trivial a formal proof can be found in the appendix for completeness

\begin{equation}
\label{eq:relationSkRkPlus1}
\Ss_{k+1}^\alpha = (A^{k+1})^{-1} (\X \ominus \Rr_{k+1}^\alpha) \cap \Ss_k^\alpha.
\end{equation}

\section{Parametric robust positively invariant sets}
\label{sec:parametricRPIsets}



interval $\I \subset \R_+$




\begin{equation}
\label{YIk}
\Y^{\I}_k := \left\{ \begin{pmatrix}
x \\
\alpha 
\end{pmatrix} \in \R^{n+1} \,\Big|\, x \in \Rr_k^\alpha,\,\,  \alpha \in \I \right\} 
\end{equation}

\begin{equation}
\label{ZIk}
\Z^{\I}_k := \left\{ \begin{pmatrix}
x \\
\alpha 
\end{pmatrix} \in \R^{n+1} \,\Big|\, x \in \Ss_k^\alpha,\,\, \alpha \in \I  \right\} 
\end{equation}

%\begin{equation}
%\label{XI}
%\Cc^{\I} := \left\{ \begin{pmatrix}
%x \\
%\alpha 
%\end{pmatrix} \in \R^{n+1} \,\Big|\, x \in \X,\,\,  \alpha \in \I \right\} 
%\end{equation}



\begin{equation}
\label{eq:criticalAlpha}
\alpha^\ast :=  \sup_{\alpha} \,\alpha \quad \text{s.t.} \quad \alpha \, \Rr^1_\infty \subseteq \X
\end{equation}


\begin{thm}
Let $\lb{\alpha},\ub{\alpha}\in \R_+$ with $\lb{\alpha} < \ub{\alpha} $ and let $k \in \N_+$. Then, $\Y^{[\lb{\alpha},\ub{\alpha}]}_k$ is a compact, convex and full-dimensional set in $\R^{n+1}$.
\end{thm}

\begin{lem}
\label{lem:YalphaConvY1}
Let $\lb{\alpha},\ub{\alpha}\in \R_+$ with $\lb{\alpha} < \ub{\alpha} $ and let $k \in \N_+$. Then,
$
\Y^{[\lb{\alpha},\ub{\alpha}]}_k = \conv \left\{ 
\lb{\alpha}\,\Y^{1}_k , 
\ub{\alpha}\,\Y^{1}_k 
\right\}.
$
\end{lem}

\begin{proof}
Consider any $\alpha \in \R_+$ and note that
\begin{equation}
\label{eq:RkAlphaRk1}
\Rr_k^\alpha= \bigoplus_{j=0}^{k-1} A^j \,\W^\alpha = \alpha \bigoplus_{j=0}^{k-1} A^j \,\W^\ast = \alpha \, \Rr_k^1.
\end{equation}
We consequently have $\Y^{\alpha}_k =\alpha\,\Y^{1}_k $, which immediately proves the claim.
\end{proof}



\begin{proof}
The sets $\Rr_k^{1}$ are C-sets in $\R^n$ for every $k \in \N_+$. This observation in combination with Lem.~\ref{lem:YalphaConvY1} proves the claim.
\end{proof}

\begin{thm}
Let $\lb{\alpha},\ub{\alpha}\in \R_+$ with $\lb{\alpha} < \ub{\alpha} \leq \alpha^\ast$ and let $k \in \N$. Then, $\Z^{[\lb{\alpha},\ub{\alpha}]}_k$ is a compact, convex and full-dimensional set in $\R^{n+1}$.
\end{thm}

\begin{proof}
The claim obviously holds for $k=0$ since $\Z^{[\lb{\alpha},\ub{\alpha}]}_0 = \X \times [\lb{\alpha},\ub{\alpha}]$ and since $\X$ is a C-set in $\R^n$ by assumption.
For $k \in \N_+$, first note that $\Ss_k^{\alpha}$ is a compact, convex and full-dimensional set in $\R^n$ for every $\alpha \in [\lb{\alpha},\ub{\alpha}]$ (due to $0<\lb{\alpha} < \ub{\alpha} \leq \alpha^\ast$). Since $[\lb{\alpha},\ub{\alpha}]$ is a compact and full-dimensional set in $\R$, it is clear that $\Z^{[\lb{\alpha},\ub{\alpha}]}_k$ is a compact and full-dimensional set in $\R^{n+1}$. It remains to show that $\Z^{[\lb{\alpha},\ub{\alpha}]}_k$ is convex. By definition, this is the case if
\begin{equation}
\label{eq:zHat}
\hat{z}:=\xi z_1+(1-\xi) z_2 \in \Z^{[\lb{\alpha},\ub{\alpha}]}_k.
\end{equation}
for every $z_1,z_2 \in \Z^{[\lb{\alpha},\ub{\alpha}]}_k$ and $\xi \in [0,1]$. Clearly $z_1$ and $z_2$ can be decomposed as
$$
z_1 = \begin{pmatrix}
x_1 \\
\alpha_1
\end{pmatrix},
\quad 
z_2 = \begin{pmatrix}
x_2 \\
\alpha_2
\end{pmatrix}
 \quad \text{and} \quad 
\hat{z} = \begin{pmatrix}
\hat{x} \\
\hat{\alpha}
\end{pmatrix}
$$
with $\alpha_1,\alpha_2,\hat{\alpha} \in \R_+$. We have $x_1 \in \Ss_k^{\alpha_1}$ and $x_2 \in \Ss_k^{\alpha_2}$ by construction.
Analogously, \eqref{eq:zHat} holds if (and only if) $\hat{x} \in \Ss_k^{\hat{\alpha}}$. According to Lem.~\ref{lem:xInSkAlpha}, $\hat{x} \in \Ss_k^{\hat{\alpha}}$ if and only if
$$
A^j \hat{x} + \sum_{i=0}^{j-1} A^{j-1-i} \hat{w}(i)  \in \X
$$
for every $j \in \N_{[1,k]}$ and every disturbance sequence $\hat{w}(0),\dots, \hat{w}(k-1) \in \W^{\hat{\alpha}}$. Consider any such sequence and define
\begin{equation}
\label{eq:w12}
w_1(i):= \frac{\alpha_1}{\hat{\alpha}} \,\hat{w}(i)
\quad \text{and} \quad 
w_2(i):= \frac{\alpha_2}{\hat{\alpha}} \,\hat{w}(i)
\end{equation}
for every $i \in \N_{[0,k-1]}$. By definition of $\W^\alpha$, it is easy to see that
$$
w_1(0),..., w_1(k-1) \in \W^{\alpha_1}
\,\, \text{and} \,\,
w_2(0),..., w_2(k-1) \in \W^{\alpha_2}.
$$
Due to $x_1 \in \Ss_k^{\alpha_1}$ and $x_2 \in \Ss_k^{\alpha_2}$, we thus have
\begin{align}
\label{eq:x1Sequence}
A^j \,x_1 + \sum_{i=0}^{j-1} A^{j-1-i} \,w_1(i)  &\in \X  \quad \text{and}\\
\label{eq:x2Sequence}
A^j \,x_2 + \sum_{i=0}^{j-1} A^{j-1-i} \,w_2(i)  &\in \X
\end{align}
for every $j \in \N_{[1,k]}$  according to Lem.~\ref{lem:xInSkAlpha}.
Now, \eqref{eq:zHat} obviously yields
\begin{align}
\label{eq:xHat}
\hat{x} &=  \xi \,x_1 + (1-\xi) \,x_2 \quad \text{and}\\
\label{eq:alphaHat}
\hat{\alpha} &= \xi \,\alpha_1 + (1-\xi) \,\alpha_2.
\end{align}
Moreover, we find 
\begin{equation}
\label{eq:xiW12}
\xi w_1(i) + (1-\xi) \,w_2(i) = \frac{\xi \alpha_1 + (1-\xi) \alpha_2}{\hat{\alpha}} \, \hat{w}(i) =\! \hat{w}(i)\!
\end{equation}
due to~\eqref{eq:w12} and \eqref{eq:alphaHat}, respectively.
Equations \eqref{eq:x1Sequence}, \eqref{eq:x1Sequence}, and \eqref{eq:xiW12} in combination with convexity of $\X$ finally prove the second relation in~\eqref{eq:zHat}.
%choose any $\alpha_1,\alpha_2 \in [\lb{\alpha},\ub{\alpha}]$ with $\alpha_1 \leq \alpha_2$, $x_1 \in \Ss_k^{\alpha_1}$, and $x_2 \in \Ss_k^{\alpha_2}$. Let $\xi \in [0,1]$ and consider $\hat{x} = \xi \,x_1 + (1-\xi) \, x_2$ and $\hat{\alpha}=\xi \alpha_1 + (1-\xi) \, \alpha_2$. Then $\hat{x} \in \Ss_k^{\hat{\alpha}}$ and consequently 
%$$
%\begin{pmatrix}
%\hat{x} \\
%\hat{\alpha}
%\end{pmatrix} \in \Z^{[\lb{\alpha},\ub{\alpha}]}_k.
%$$
%To see this, assume that $\hat{x} \notin \Ss_k^{\hat{\alpha}}$
%
%$\Ss_k^{\alpha_1} \supseteq \Ss_k^{\hat{\alpha}} \supseteq  \Ss_k^{\alpha_2}$. Thus $x_2 \in \Ss_k^{\hat{\alpha}}$.
%Having $\hat{x} \notin \Ss_k^{\hat{\alpha}}$ consequently implies $x_1  \notin \Ss_k^{\hat{\alpha}}$ since $\Ss_k^{\hat{\alpha}}$ is convex.
\end{proof}

the limit $\Z^{[\lb{\alpha},\ub{\alpha}]}_\infty $ is finitely determined, i.e., there exists a finite $N \in \N$ such that

As a preperation, we introduce the scaling matrix
 $$
 E^\epsilon := \begin{pmatrix}
 (1+\epsilon) I_n & 0 \\ 0 & 1
 \end{pmatrix} \in \R^{(n+1) \times (n+1)}
 $$ 
for some $\epsilon \in \R_+$ and the projection matrix
$$
P:=\begin{pmatrix}
  I_n & 0 
 \end{pmatrix} \in \R^{n \times (n+1)}.
 $$

\begin{thm}
\label{thm:outerApproxYinfI}
Let Assum.~\ref{assum:AXW} be satisfied, let $\lb{\alpha},\ub{\alpha}\in \R_+$ with $\lb{\alpha} < \ub{\alpha}< \alpha^\ast$, and let $\epsilon \in \R_+$.
 Assume $M \in \N$ is such that
 \begin{equation}
\label{eq:conditionYEpsApproximation}
A^{M} \W^\ast \subseteq \mu \,\W^\ast \quad \text{with} \quad \mu:=\frac{\epsilon}{1+\epsilon}.
\end{equation}
 Then,
$\Y^{[\lb{\alpha},\ub{\alpha}]}_\infty \subseteq E^\epsilon\Y^{[\lb{\alpha},\ub{\alpha}]}_M$.
 Moreover, if
 \begin{equation}
\label{eq:boundOnEpsilon}
\epsilon \leq \frac{\alpha^\ast}{\ub{\alpha}}-1,
\end{equation}
 $P E^\epsilon \Y^{\alpha}_M$ is an RPI set for~\eqref{eq:system} with~\eqref{eq:constraints} for every $\alpha \in [\lb{\alpha},\ub{\alpha}]$. 
\end{thm}

\begin{proof}
According to \cite[Thm.~1]{Rakovic2005}, \eqref{eq:conditionYEpsApproximation} yields
\begin{equation}
\label{eq:RInfty1EpsApproximation}
\Rr_\infty^1 \subseteq \frac{1}{1-\mu} \Rr_M^1 = (1+\epsilon) \, \Rr_M^1.
\end{equation}
Following~\eqref{eq:RkAlphaRk1}, relation~\eqref{eq:RInfty1EpsApproximation} implies
$$
\Rr_\infty^\alpha \subseteq  (1+\epsilon) \, \Rr_M^\alpha \quad \text{for every} \quad \alpha \in [\lb{\alpha},\ub{\alpha}].
$$
Obviously, $\Rr_\infty^\alpha \subseteq  (1+\epsilon) \, \Rr_M^\alpha $ can equivalently be stated as $
\Rr_\infty^\alpha \subseteq  (1+\epsilon) \,I_p\, \Rr_M^\alpha$ 
which proves $\Y^{[\lb{\alpha},\ub{\alpha}]}_\infty \subseteq E^\epsilon\Y^{[\lb{\alpha},\ub{\alpha}]}_M$. To see that \eqref{eq:boundOnEpsilon} implies that $P E^\epsilon \Y^{\alpha}_M$ is an RPI set  for every $\alpha \in [\lb{\alpha},\ub{\alpha}]$, first note that $P E^\epsilon \Y^{\alpha}_M = (1+\epsilon) \Rr_M^\alpha$. Clearly, $(1+\epsilon) \Rr_M^\alpha \subseteq \X$ due to $\alpha \leq \ub{\alpha} < \alpha^\ast$ and~\eqref{eq:boundOnEpsilon}. Moreover, according to the proof of \cite[Thm.~1]{Rakovic2005}, \eqref{eq:conditionYEpsApproximation} implies 
$$ (1+\epsilon) A \, \Rr_M^\alpha \oplus \W^\alpha \subseteq (1+\epsilon) \Rr_M^\alpha,
$$
which proves that $(1+\epsilon) \Rr_M^\alpha$ is an RPI set  for~\eqref{eq:system} with~\eqref{eq:constraints} according to Def.~\ref{def:RPI}.
\end{proof}






\begin{thm}
Let Assum.~\ref{assum:AXW} be satisfied and let $\lb{\alpha},\ub{\alpha}\in \R_+$ with $\lb{\alpha} < \ub{\alpha} < \alpha^\ast$. Assume $N \in \N$ is such that
 \begin{equation}
\label{eq:conditionZFinitelyDetermined}
A^{N+1} \X \subseteq \eta \,\X \quad \text{with} \quad \eta:=1-\frac{\ub{\alpha}}{\alpha^\ast}.
\end{equation}
 Then,  $\Z^{[\lb{\alpha},\ub{\alpha}]}_\infty = \Z^{[\lb{\alpha},\ub{\alpha}]}_N $. Moreover,   $P \Z^{\alpha}_N$ is the MRPI set for~\eqref{eq:system} with~\eqref{eq:constraints} for every $\alpha \in [\lb{\alpha},\ub{\alpha}]$. 
\end{thm}



%\begin{thm}
%Let $\lb{\alpha},\ub{\alpha}\in \R_+$ with $\lb{\alpha} < \ub{\alpha} < \alpha^\ast$. Then, the limit $\Z^{[\lb{\alpha},\ub{\alpha}]}_\infty $ is finitely determined, i.e., there exists a finite $N \in \N$ such that $\Z^{[\lb{\alpha},\ub{\alpha}]}_\infty = \Z^{[\lb{\alpha},\ub{\alpha}]}_N $.
%\end{thm}

\begin{proof}
We have $\Z^{[\lb{\alpha},\ub{\alpha}]}_\infty = \Z^{[\lb{\alpha},\ub{\alpha}]}_N $ if (and only if) 
\begin{equation}
\label{eq:ZNPlus1EqualsZN}
\Z^{[\lb{\alpha},\ub{\alpha}]}_{N+1}= \Z^{[\lb{\alpha},\ub{\alpha}]}_N.
\end{equation}
Condition~\eqref{eq:ZNPlus1EqualsZN} holds 
 if (and only if) 
\begin{equation}
\label{eq:SNPlus1EqualsSN}
 \Ss_{N+1}^\alpha = \Ss_{N}^\alpha \quad \text{for every} \quad \alpha \in [\lb{\alpha},\ub{\alpha}].
\end{equation}
  Now, from relation~\eqref{eq:relationSkRkPlus1}, it is easy to see that~\eqref{eq:SNPlus1EqualsSN} holds if (and only if)
 \begin{equation}
\label{eq:exactConditionZFinitelyDetermined}
  A^{N+1} \Ss_N^\alpha \subseteq \X \ominus \Rr_{N+1}^\alpha \quad \text{for every} \quad \alpha \in [\lb{\alpha},\ub{\alpha}].
\end{equation}
We next show that~\eqref{eq:exactConditionZFinitelyDetermined} holds if 
 \begin{equation}
\label{eq:auxiliaryConditionZFinitelyDetermined}
A^{N+1} \X \subseteq \X \ominus \Rr^{\ub{\alpha}}_\infty.
\end{equation}
To see this, note that  $A^{N+1} \Ss_N^\alpha \subseteq  A^{N+1} \Ss_N^{\lb{\alpha}} \subseteq A^{N+1} \X$ and $ \X \ominus \Rr_{N+1}^\alpha \supseteq \X \ominus \Rr_{N+1}^{\ub{\alpha}} \supseteq \X \ominus \Rr_{\infty}^\alpha $ for every $ \alpha \in [\lb{\alpha},\ub{\alpha}]$.
The r.h.s.~in \eqref{eq:auxiliaryConditionZFinitelyDetermined} can be further underestimated. In fact, due to $0<\ub{\alpha} < \alpha^\ast$, we have 
$$
\X \ominus \Rr^{\ub{\alpha}}_\infty \supseteq \eta \,\X  \quad \text{with} \quad \eta:=1-\frac{\ub{\alpha}}{\alpha^\ast} \in (0,1)
$$
according to~\eqref{eq:criticalAlpha} and \cite[Rem.~2.1]{Kolmanovsky1998}. Thus, \eqref{eq:auxiliaryConditionZFinitelyDetermined} and consequently~\eqref{eq:ZNPlus1EqualsZN} holds, if~\eqref{eq:conditionZFinitelyDetermined} is fulfilled.
\end{proof}

%It is important to note that the set $\Gamma\,\Y^{[\lb{\alpha},\ub{\alpha}]}_M $ from Thm.~\ref{thm:outerApproxYinfI} is
%
%\begin{align}
%\label{eq:sequenceTk}
%\T_{k+1}^\alpha &:= A^{-1} (\T_k^\alpha \ominus \W^\alpha) \cap \X \quad  \text{with} \\  
%\T_{0}^\alpha &:=\X \cap (1+\epsilon) \,\Rr_M^\alpha ,
%\end{align}
%
%
%$$
%\M^{\I}_k := \left\{ \begin{pmatrix}
%x \\
%\alpha 
%\end{pmatrix} \in \R^{n+1} \,\Big|\, x \in \T_k^\alpha,\,\, \alpha \in \I  \right\} 
%$$


\begin{rem}
Obviously, $M \in \N$ and $N \in \N$ satisfying~\eqref{eq:conditionYEpsApproximation} and~\eqref{eq:conditionZFinitelyDetermined}, respectively, can always be found since both $\mu$ and $\eta$ lie in the interval $(0,1)$, since the eigenvalues of $A$ are strictly stable, and since $\X$ is a C-set according to Assum.~\ref{assum:AXW}. Assume, for example, that $\lb{r}_w,\ub{r}_w \in \R_+$ denote the radii of a ball that is contained in $\W^\ast$ and a ball that contains $\W^\ast$, respectively. Then, \eqref{eq:conditionYEpsApproximation} holds if 
\begin{equation}
\label{eq:MBoundApprox}
\|A^M\|_2 \leq \frac{\mu \,\lb{r}_w}{\ub{r}_w}. 
\end{equation}
\end{rem}






\subsection{Approximation of the critical scaling}

Accurate approximation of critical scaling $\alpha^\ast$ for a user-defined error bound $\epsilon \in \R_+$.

\begin{thm}
Let Assum.~\ref{assum:AXW} be satisfied and let $\epsilon \in \R_+$.
 Assume $M \in \N$ is such that~\eqref{eq:conditionYEpsApproximation} holds and define
 \begin{equation}
\label{eq:criticalAlphaApprox}
\hat{\alpha}^\ast :=  \max_{\alpha} \,\alpha \quad \text{s.t.} \quad \alpha \, (1+\epsilon)\, \Rr^1_M \subseteq \X.
\end{equation}
Then, $\hat{\alpha}^\ast \leq \alpha^\ast \leq (1+\epsilon)\, \hat{\alpha}^\ast$.
\end{thm}

\begin{proof}
Analogously to the proof of Thm.~\ref{thm:outerApproxYinfI}, satisfaction of \eqref{eq:conditionYEpsApproximation} yields
\eqref{eq:RInfty1EpsApproximation}. In combination with~\eqref{eq:criticalAlphaApprox}, we thus obtain $ \hat{\alpha}^\ast\,\Rr_\infty^1 \subseteq  \hat{\alpha}^\ast (1+\epsilon) \, \Rr_M^1 \subseteq \X$ which implies $\hat{\alpha}^\ast \leq \alpha^\ast $ according to~\eqref{eq:criticalAlpha}. Similarly, we have $\alpha^\ast \,\Rr_M^1 \subseteq \alpha^\ast \,\Rr_\infty^1 \subseteq \X$ by construction and thus $\alpha^\ast \leq (1+\epsilon)\, \hat{\alpha}^\ast$.
\end{proof}


\section{Computation and numerical examples}

In Sect.~\ref{sec:parametricRPIsets}, we discussed properties of parametric RPI sets and the approximation of the critical scaling $\alpha^\ast$ for general C-sets $\X$ and $\W^\ast$.
However, the actual computation of the sets $\Y^{[\lb{\alpha},\ub{\alpha}]}_k$ and $\Z^{[\lb{\alpha},\ub{\alpha}]}_k$ and the approximation of $\alpha^\ast$ is demanding for this general setup. 
We thus consider polytopic sets $\X$ and $\W^\ast$ to illustrate the  findings in Sect.~\ref{sec:parametricRPIsets}. 
In this case, it is well-known that the sets $\Rr_k^\alpha$ and $\Ss_k^\alpha$ are  polytopes as well.
As for $\Rr_k^\alpha$, this observation immediately follows from~\eqref{eq:RkAlphaRk1}. Regarding $\Ss_k^\alpha$, we first
assume that the state constraints $\X$ are given by
$$
\X = \{ x \in \R^n \,|\, H_x \, x \leq \boldsymbol{1} \}
$$
with $H \in \R^{L \times n}$ and $d \in \R^L$.
Then,  the sets $\Ss_k^\alpha$ can be expressed as
\begin{equation}
\label{eq:SkPolytope}
\!\,\Ss_k^\alpha \!=\! \bigg\{ \!x \!\in\! \R^n \bigg| H A^j x \leq d -\! \sum_{i=0}^{j-1} \!h_{\W^\alpha}\!\,(H A^i), \, \forall j \! \in \! \N_{[0,k]} \!\bigg\}\!\!\!\!\!
\end{equation}
according to \cite[p. 342]{Kolmanovsky1998}.
Thereby, the term $h_{\W^\alpha} (H A^i)$ can be computed by solving $L$ linear programs. 
Having polytopic sets $\Rr_k^\alpha$ and taking Lem.~\ref{lem:YalphaConvY1} into account, immediately implies that $
\Y^{[\lb{\alpha},\ub{\alpha}]}_k$ is a polytope.
To see that $
\Z^{[\lb{\alpha},\ub{\alpha}]}_k$ is polytopic as well, first note that
\begin{equation}
\label{eq:hWalpha}
h_{\W^\alpha}(c) = \sup_{w \in \W^\alpha}  c \,w = \alpha \sup_{w^\ast \in \W^\ast}  c \, w^\ast = \alpha \, h_{\W^\ast}(c)
\end{equation}
for any $c \in \R^{1 \times n}$.
Using~\eqref{eq:hWalpha} in~\eqref{eq:SkPolytope} and substituting the result in~\eqref{ZIk} yields
\begin{align}
\nonumber
\!\!\Z_k^{[\lb{\alpha},\ub{\alpha}]}= \bigg \{ z \in \R^{n+1} \,\bigg|\, \begin{pmatrix} H A^j & \sum \limits_{i=0}^{j-1} h_{\W^\ast}(H A^i)  \end{pmatrix} z \leq d, \!\!\!\!\!\!\\
\label{eq:ZkPolytope}
\lb{\alpha} \leq z_{n+1} \leq \ub{\alpha},  \, \forall  j \in \N_{[0,k]} \bigg\} \quad  \!\!\!\!\!
\end{align}
after some basic manipulations.
Clearly, $\Z_k^{[\lb{\alpha},\ub{\alpha}]}$ as in~\eqref{eq:ZkPolytope} is a polytope. 
This observation is discussed in more detail in~\cite{Schaich2015}.

Polytopic sets $\X$ and $\W^\ast$ do not only simplify the computation of $\Y^{[\lb{\alpha},\ub{\alpha}]}_k$ and $
\Z^{[\lb{\alpha},\ub{\alpha}]}_k$.
They also allow to efficiently and accurately compute $M$, $N$, and $\hat{\alpha}^\ast$ as in such that~\eqref{eq:conditionYEpsApproximation}, \eqref{eq:conditionZFinitelyDetermined}, and~\eqref{eq:criticalAlphaApprox}. 


In fact, as discussed in \cite[Eq.~(10)]{Rakovic2005},
Eq.~\eqref{eq:conditionYEpsApproximation} holds if and only if 
$$
\max_{i \in \N_{[1,l_w]}} h_{\W^\ast}((e_i^T H_w A^M)^T) \leq \mu 
$$

In fact, for a polytopic set $\W^\ast$ it is not necessary to 

{eq:MBoundApprox}


\subsection{Example 1}

We first consider system~\eqref{eq:system} with 
$A=0.5$
and the constraints $\X=[-2,2]$ and $\W^\ast=[-1,1]$. For this simple example, which was also analyzed in \cite[Exmp.~6.10]{Blanchini2008} (without state constraints), the sets $\Rr_k^\alpha$ and $\Ss_k^\alpha$ can be explicitly stated. 
In fact, it is easy to see that we 

$$
\rho_k^\alpha:=\alpha \, (2 - 0.5^k)
$$

$\alpha^\ast=1$

$\hat{\alpha}^\ast=...$ for $\epsilon=0.1$, $\mu=\frac{1}{11}$

$\lb{\alpha}=0.2$,
$\ub{\alpha}=0.8$

Since
$$
\epsilon \leq  \frac{\hat{\alpha}^\ast}{\ub{\alpha}}-1 = .. \leq \frac{\alpha^\ast}{\ub{\alpha}}-1 = 0.25,
$$

$$
\hat{\eta}:= 1 - \frac{\ub{\alpha}}{\hat{\alpha}^\ast} = ... \leq 
$$

\subsection{Example 2}

 \cite[Sect.~4.1]{Mayne2005} 

$\begin{pmatrix}
\blind{+}0.6696 & \blind{+}0.3369  \\ -0.6609 & -0.3261
\end{pmatrix}$  


  $\begin{array}{rl}
-4\!\!\! &\leq x_2 \leq 2 \\
-1\!\!\! &\leq K x \leq 1
\end{array}$  

 $\|w\|_\infty \leq 0.1$  
 
  $3.362391$    $3.362728$
  
 
 $\lb{\alpha}=0.5$
 
 $\ub{\alpha}=2.5$

%!TEX root = 2016_CDC.tex
\section{Numerical examples}

In this section we illustrate the obtained results with simple example systems.
%
First we consider the scalar system presented in~\cite[Exmp.~6.10]{Blanchini2008}:
%
\[
	x(k+1) = \frac{1}{2}x(k) + w
\]
%
where the state is constrained to the interval~$\X=[-2,2]$ and the disturbance has the nominal amplitude~$\W^\ast=[-1,1]$.
%
In this case the critical alpha can be explicitly calculated to be~$\alpha^\ast=1$ and both $\mathcal Z_\infty^{[0,\alpha^\ast]}$ and $\mathcal Y_\infty^{[0,\alpha^\ast]}$ can be explicitly determined.
%
The graph~$\mathcal Z^{[\frac{1}{2},1]}_\infty$ is illustrated in Figure~\ref{fig:first:example}.


The second system we consider is the two-dimensional example presented in~\cite[Sect.~4.1]{Mayne2005}:
%
\[
	x(k+1) = \left(\begin{array}{cc} 1 &1 \\ 0& 1\end{array}\right) x(k) + \left(\begin{array}{c}\frac{1}{2}\\1\end{array}\right) u(k) + w(k)
\]
%
where the input is given by the Riccati controller $u = Kx$  with $Q = I$ and $R=0.01$.
%
The constraints for this system are given by $\mathcal X = \{x\in\mathbb R^2\vert x_2\leq2,-1\leq Kx\leq 1\}$ and $\mathcal W^\ast = [-\frac{1}{10},\frac{1}{10}]\times [-\frac{1}{10},\frac{1}{10}]$.
%
For this setup the approximation of $\alpha^\ast$ leads to the lower and upper bounds $\underline\alpha = 3.362391$ and $\bar\alpha=3.362728$ respectively.
%
\begin{figure}
\centering
\begin{tikzpicture}
\draw[-latex'] (0,0) node[below] {$0$} -- (0,2) node[right] {$\alpha$};
\draw[-latex'] (-2.2,0) -- (2.2,0) node[right] {$x$};
\draw (.1,1) -- (-.1,1) node[above right] {$\alpha^\ast$};
\draw (-2,1) -- (0,0) -- (2,1);
\fill[pattern = north east lines, opacity=0.8] (-2,1) -- (0,0) -- (2,1) -- cycle;
\draw (2,.1) -- (2,-.1) node[below] {$2$};
\draw (-2,.1) -- (-2,-.1) node[below] {$-2$};
\end{tikzpicture}
\caption{$\mathcal Y_\infty^{[0,\alpha^\ast]}$ for $x(k+1) = \frac{1}{2}x(k) + w$.}\label{fig:first:example:mRPI}
\end{figure}
%
\begin{figure}\centering
\begin{tikzpicture}
	\draw[-latex'] (0,0) node[below] {$0$} -- (0,2) node[right] {$\alpha$};
	\draw[-latex'] (-2.2,0) -- (2.2,0) node[right] {$x$}; 
	\fill[pattern = north east lines, opacity=0.8] (-2,0) -- (2,0) node[below] {$2$} -- (2,1) -- (-2,1) -- cycle;
	\draw (2,0) -- (2,1);
	\draw (.1,1) -- (-.1,1) node[above right] {$\alpha^\ast$};
	\draw (-2,0) node[below] {$-2$} -- (-2,1);
\end{tikzpicture}
\caption{$\mathcal Z_\infty^{[0,\alpha^\ast]}$ for the $x(k+1) = \frac{1}{2}x(k) + w$.}\label{fig:first:example:MRPI}
\end{figure}

\begin{figure}
\tdplotsetmaincoords{80}{-30}
\centering
\begin{tikzpicture}[tdplot_main_coords]
\draw[-latex'] (0,0,0) -- (0,0,3.5) node[right] {$\alpha$};
\draw[-latex'] (0,0,0) -- (0,2.1,0) node[below] {$x_2$};
\draw[-latex'] (0,0,0) -- (3.5,0,0) node[below] {$x_1$};
\draw (2.5562,-0.5198,0) -- (-2.5000, 2.0000, 0) -- (-3.2716, 2.0000, 0) -- (-2.5562, 0.5198, 0) -- (3.5165, -2.5066,0) -- cycle;
\draw (-2.5000, 2.0000, 0) -- (-2.5000, 2.0000, 1.6849) -- (-3.2716, 2.0000, 0);
\draw (-2.5000, 2.0000, 1.6849) -- (-1.4878, 1.4956, 3.3627);
\draw (-2.5562, 0.5198, 0) -- (-0.5275, -0.4912, 3.3627);
\draw (3.5165, -2.5066,0) -- (1.4878, -1.4956, 3.3627);
\draw (2.5562,-0.5198,0) -- (0.5275,0.4912,3.3627);
\draw (1.4878, -1.4956, 3.3627) -- (0.5275,0.4912,3.3627) -- (-1.4878, 1.4956, 3.3627) -- (-0.5275, -0.4912, 3.3627) -- cycle;
\end{tikzpicture}
\caption{$\mathcal Z_\infty^{[0,\underline\alpha]}$}\label{fig:second:example}
\end{figure}

\begin{figure}\centering
\begin{tikzpicture}
\draw[-latex'] (0,0) -- (0,2) node[right] {$x_2$};
\draw[-latex'] (0,0) -- (2.5,0) node[below] {$x_1$};
\draw (1.9529,   -0.2191) -- (-2.5000,    2.0000) -- (-2.8137,    2.0000) -- (-1.9529,    0.2191) -- (2.9132,   -2.2059) -- cycle; % \alpha = 1
\draw (1.5397,   -0.0132) -- (-2.5000,    2.0000) -- (-2.5000,    2.0000) -- (-1.5397,    0.0132) -- (2.5000,   -2.0000) -- cycle; %\alpha = 1.6849
\draw[densely dotted] (2.0615,   -1.0928) -- (1.5397,   -0.0132) -- (-0.8800,    1.1927) -- (-2.2755,    1.8881) -- (-2.4459,    1.8881) -- (-1.5397,    0.0132) -- (1.1045,   -1.3046) -- (2.5000,   -2.0000) -- cycle; % interpolated
\draw (2.0082,   -1.7549) -- (1.0480,    0.2318) -- (-2.0082,    1.7549) -- (-1.0480,   -0.2318) -- cycle; %\alpha = 2.5
\end{tikzpicture}
\end{figure}

\section{Conclusion and outlook}
\label{sec:conclusionOutlook}






\bibliographystyle{ieeetr}
%\bibliographystyle{plain}        % Include this if you use bibtex 
\bibliography{msdLiterature} 

\appendix

\begin{lem}
\label{lem:PmaxPmin}
Let Assum.~\ref{assum:AXW} be satisfied, let $\alpha \in \R_+$, and let $k \in \N$. Then, the set $\Ss_k^\alpha$ defined by the sequence~\eqref{eq:sequenceSk} can be expressed as in~\eqref{eq:relationSkRk}.
\end{lem}

\begin{proof}
We prove the claim by induction. Relation~\eqref{eq:relationSkRk} obviously holds for $k = 0$ since obtain $\Ss_0^\alpha = \X$ as in~\eqref{eq:sequenceSk}. It remains to show that~\eqref{eq:relationSkRk} implies
\begin{equation}
\label{eq:inductionStep}
\Ss_{k+1}^\alpha = \bigcap_{j=0}^{k+1} (A^{j})^{-1} (\X \ominus \Rr_{j}^\alpha).
\end{equation}
To this end, first note that~\eqref{eq:sequenceSk}  in combination with \eqref{eq:relationSkRk} yields
\begin{equation}
\label{eq:SkPlus1Rewritten}
\Ss_{k+1}^\alpha = A^{-1} \! \left( \left(\bigcap_{j=0}^k (A^{j})^{-1} (\X \ominus \Rr_{j}^\alpha) \right) \ominus \W^\alpha\right) \cap \X
\end{equation}
Now, the r.h.s.~in~\eqref{eq:SkPlus1Rewritten} can be rewritten as 
\begin{align}
\nonumber
&A^{-1} \!\left( \bigcap_{j=0}^k (A^{j})^{-1} (\X \ominus \Rr_{j}^\alpha \ominus A^j  \W^\alpha )  \right) \cap \X \\
\nonumber
&= \left( \bigcap_{j=0}^k (A^{j+1})^{-1} (\X \ominus \Rr_{j}^\alpha \ominus A^j  \W^\alpha )  \right) \cap \X \\
\nonumber
&=  \left( \bigcap_{j=1}^{k+1} (A^{j})^{-1} (\X \ominus \Rr_{j+1}^\alpha \ominus A^{j+1}  \W^\alpha )  \right) \cap \X 
\end{align}
Since $\Rr_{j+1}^\alpha \ominus A^{j+1}  \W^\alpha = \Rr_{j}^\alpha$ for every $j \in \N$ due to convexity and compactness of $\Rr_{j+1}^\alpha$, $A^{j+1}  \W^\alpha$, and $\Rr_{j}^\alpha$  (see \cite[p. 325]{Kolmanovsky1998}), we finally obtain
\begin{equation}
\label{eq:finalLine}
\Ss_{k+1}^\alpha = \left( \bigcap_{j=1}^{k+1} (A^{j})^{-1} (\X \ominus \Rr_{j}^\alpha )  \right) \cap \X 
\end{equation}
Clearly, \eqref{eq:inductionStep} and~\eqref{eq:finalLine} are equivalent.
\end{proof}



\end{document}
