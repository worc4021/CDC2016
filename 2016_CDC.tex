%%%%%%%%%%%%%%%%%%%%%%%%%%%%%%%%%%%%%%%%%%%%%%%%%%%%%%%%%%%%%%%%%%%%%%%%%%%%%%%%
%
%        1         2         3         4         5         6         7         8

%\documentclass[letterpaper, 10 pt, conference]{ieeeconf}  % Comment this line out
                                                          % if you need a4paper
%\documentclass[a4paper, 10pt, conference]{ieeeconf}      % Use this line for a4
%\documentclass[10pt, conference]{ieeeconf}                                                    % paper

\documentclass[letterpaper, 10 pt, conference]{ieeeconf}

\IEEEoverridecommandlockouts                              % This command is only
                                                          % needed if you want to
                                                          % use the \thanks command
\overrideIEEEmargins
% See the \addtolength command later in the file to balance the column lengths
% on the last page of the document




\usepackage{epsfig} % for postscript graphics files
\usepackage{amsmath} % assumes amsmath package installed
\usepackage{amssymb}  % assumes amsmath package installed
\usepackage{psfrag}
\usepackage{color}
\usepackage[usenames,dvipsnames,table]{xcolor}
\usepackage{units}
\usepackage[top=54pt, left=54pt, right=54pt, bottom=54pt, paper=letterpaper]{geometry}
\usepackage{paralist}
\usepackage[table]{xcolor}
\usepackage[noadjust]{cite}
\usepackage{epstopdf}
\usepackage{booktabs}
\usepackage{tikz}
\usetikzlibrary{arrows,positioning,patterns,decorations.pathreplacing}
\usepackage{tikz-3dplot}
\usepackage[labelformat=empty]{subfig}

\newcommand{\subparagraph}{}

%\usepackage{cite}
\usepackage{titlesec}
%

\usepackage{accents}
\newlength{\dhatheight}
\newcommand{\doublehat}[1]{%
    \settoheight{\dhatheight}{\ensuremath{\hat{#1}}}%
    \addtolength{\dhatheight}{-0.25ex}%
    \hat{\vphantom{\rule{1pt}{\dhatheight}}%
    \smash{\hat{#1}}}}

%\titlespacing{\section}{3em}{2pt}{2pt}
%\titlespacing{\subsection}{1em}{1pt}{1pt}
%\titlespacing{\subsubsection}{1em}{1pt}{1pt}

%\abovedisplayshortskip=1pt
%\belowdisplayshortskip=1pt
%\abovedisplayskip=1pt
%\belowdisplayskip=1pt





%% own commands and environments
 %--------------------------------------
 
%% own macros
\newtheorem{thm}{Theorem}
\newtheorem{cor}[thm]{Corollary}
\newtheorem{lem}[thm]{Lemma}
\newtheorem{assum}{Assumption}
\newtheorem{prop}[thm]{Proposition}
\newtheorem{defn}{Definition}
\newtheorem{exmp}{Example}
\newtheorem{rem}{Remark}
\newtheorem{alg}{Algorithm}

\newcommand{\R}{\mathbb{R}} 
\newcommand{\N}{\mathbb{N}} 
\newcommand{\C}{\mathbb{C}}

\newcommand{\A}{\mathcal{A}}
\newcommand{\X}{\mathcal{X}}
\newcommand{\B}{\mathcal{B}}
\newcommand{\D}{\mathcal{D}}
\newcommand{\Cc}{\mathcal{C}}
\newcommand{\Pp}{\mathcal{P}} 
\newcommand{\Ll}{\mathcal{L}} 
\newcommand{\Rr}{\mathcal{R}} 
\newcommand{\Ss}{\mathcal{S}}
\newcommand{\W}{\mathcal{W}} 
\newcommand{\T}{\mathcal{T}}
\newcommand{\I}{\mathcal{I}}  
\newcommand{\V}{\mathcal{V}} 
\newcommand{\U}{\mathcal{U}} 
\newcommand{\Q}{\mathcal{Q}} 
\newcommand{\Y}{\mathcal{Y}}
\newcommand{\Z}{\mathcal{Z}}
\newcommand{\M}{\mathcal{M}}

\newcommand{\conv}{\mathrm{conv}}

\newcommand{\lb}[1]{\underline{#1}} 
\newcommand{\ub}[1]{\overline{#1}} 
\newcommand{\blind}[1]{\textcolor{white}{#1}}

\title{\bf Parametric robust positively  invariant sets  \\
for linear systems with scaled disturbances}

%\title{\bf Similarities and disparities among \\
%nonlinear MPC schemes with guaranteed stability}


%\author{ \parbox{3 in}{\centering Huibert Kwakernaak*
%         \thanks{*Use the $\backslash$thanks command to put information here}\\
%         Faculty of Electrical Engineering, Mathematics and Computer Science\\
%         University of Twente\\
%         7500 AE Enschede, The Netherlands\\
%         {\tt\small h.kwakernaak@autsubmit.com}}
%         \hspace*{ 0.5 in}
%         \parbox{3 in}{ \centering Pradeep Misra**
%         \thanks{**The footnote marks may be inserted manually}\\
%        Department of Electrical Engineering \\
%         Wright State University\\
%         Dayton, OH 45435, USA\\
%         {\tt\small pmisra@cs.wright.edu}}
%}

\author{Moritz Schulze Darup$^{\dagger}$, Rainer Manuel Schaich$^{\dagger}$, and Mark Cannon% <-this % stops a space
\thanks{*This research was partially funded by the German Research Foundation (DFG) under the grants SCHU 2094/1-1 and SCHU 2094/2-1.}%
% <-this % stops a space
\thanks{$^{\dagger}$ These authors contributed equally to this work.}% <-this % stops a space
\thanks{M. Schulze Darup, R. M. Schaich, and M. Cannon are with 
the Control Group, Department of Engineering Science,
        University of Oxford, Parks Road, Oxford OX1 3PJ, UK.
        E-mail: {\tt mark.cannon@eng.ox.ac.uk}.}%
}


\begin{document}
\maketitle
\thispagestyle{empty}
\pagestyle{empty}

%%%%%%%%%%%%%%%%%%%%%%%%%%%%%%%%%%%%%%%%%%%%%%%%%%%%%%%%%%%%%%%%%%%%%%%%%%%%%%%%
\begin{abstract}
\textcolor{red}{TBD}
\end{abstract}

%%%%%%%%%%%%%%%%%%%%%%%%%%%%%%%%%%%%%%%%%%%%%%%%%%%%%%%%%%%%%%%%%%%%%%%%%%%%%%%%
\section{Introduction}


Robust positively invariant (RPI) sets are important for performance analysis and synthesis of controllers for uncertain systems (see, e.g., \cite[Sects.~6.4 and 6.5]{Blanchini1999}).
In particular, RPI sets can be used to design robust model predictive control (MPC) schemes with guaranteed stability (see, e.g., \cite{Kouvaritakis2015,Lee1999,Mayne2006,Mayne2005}). In this paper, we address RPI sets for linear disturbed systems of the form
\begin{equation}
\label{eq:system}
x(k+1) = A \, x(k) + w(k)
\end{equation}
 with state and disturbance constraints
\begin{equation}
\label{eq:constraints}
x(k) \in \X  \quad \text{and} \quad w(k) \in \W^\alpha \quad \text{for every} \quad k \in \N.
\end{equation}
The set $\W^\alpha := \alpha \, \W^\ast$ denotes a scaled version  of a nominal disturbance set $\W^\ast$ for some scalar $\alpha>0$.
Roughly speaking, an RPI set $\Pp$ for system~\eqref{eq:system} with constraints~\eqref{eq:constraints} is such that the trajectory of the disturbed system~\eqref{eq:system} remains in $\Pp$ at all times $k\in \N$ for every initial conditions $x(0) \in \Pp$ and for all disturbances $w(k) \in \W^\alpha$ (see Def.~\ref{def:RPI} further below for a formal definition of RPI sets).
Now, in this study, we study how RPI sets for system~\eqref{eq:system} with constraints~\eqref{eq:constraints} change with the scaling $\alpha$. More precisely, we intend to simultaneously describe a family of RPI sets for a given interval $[\lb{\alpha},\ub{\alpha}]$ of scaling factors. 
Such parametric sets can, for example, be 
used to study the sensitivity of the RPI sets to changes in
the disturbance ``strength''.


Parametric RPI sets were previously studied in \cite{Schaich2015}. We extend the results in \cite{Schaich2015} in three directions.
First, we present characterizations of parametric RPI sets that do not require $\X$ and $\W^\ast$ to be polytopic as in \cite{Schaich2015}.
Second, we address parametric descriptions of minimal and maximal RPI sets while \cite{Schaich2015} is limited to the analysis of the maximal ones.
Third, we present constructive methods to avoid the occurrence of empty RPI sets. This is important since it is easy to see that large scalings $\alpha$ exclude the existence of nonempty RPI sets (see, e.g., \cite[Exmp.~4.1]{Kolmanovsky1995}, \cite[p. 114]{Kouvaritakis2015}, and \cite[Sect.~4]{Lee1999}).


For our analysis, we make the standing assumptions that all eigenvalues of the system matrix $A \in \R^{n \times n}$ are strictly stable (i.e., $| \lambda|<1$) and that the sets $\X$ and $\W^\ast$ are convex and compact sets in $\R^n$ that contain the origin as an interior point. Moreover, we only consider positive scalings $\alpha$.
Note that the assumption that $\W^\ast$ is full-dimensional is restrictive. However, the same setup was studied in \cite{Rakovic2005} and we will utilize some of the results in~\cite{Rakovic2005}. 

%In this paper, the sets $\X$ and $\W^\ast$ are assumed to be full-dimensional (see Assum.~\ref{assum:AXW} further below). While this is restrictive, the same setup was studied in \cite{Rakovic2005} % \cite{Rakovic2005,Schaich2015}.

The paper is organized as follow. We state notation in the remainder of this section. \textcolor{red}{TBD}

\subsection{Notation}

We denote positive real and natural numbers by $\R_+$ and $\N_+$, respectively. The notion $\N_{[i,k]}$ refers to $\N_{[i,k]}:=\{ j \in \N \,|\, i \leq j \leq k \}$.
Now, let $p \in \N_+$ and consider two bounded and convex sets $\U , \V \subset \R^p$ containing the origin. We frequently use the following manipulations of sets.
 The scaling of a set by some factor $\beta \in \R_+$  is defined as $\beta \,\U := \{ \beta u \in \R^p \,|\, u \in \U\} $.
Moreover, for $q \in \N_+$ and some matrices $C \in \R^{p\times p}$ and $D \in \R^{q \times p}$, we define $C^{-1} \, \U:= \{ \xi \in \R^p \,|\, C\,\xi \in \U \}$ and  $D\,\U := \{ D u \in \R^q \,|\, u \in \U \} $. Note that $C^{-1} \, \U$ is well defined even if $C$ is not invertible. The operations
\begin{align*}
\U \oplus \V &:= \{ u+v \in \R^p \,|\,  u \in \U, v \in \V\} \qquad \text{and} \\
\U \ominus \V &:= \{ \xi \in \R^p \,|\, \forall v \in \V: \xi+v \in \U\},
\end{align*}
describe the Minkowski addition and the Pontryagin difference, respectively. Finally, the support function of the set $\U$ and a row vector $c \in \R^{1 \times p}$ is defined as
$$
h_{\U}(c) := \sup_{u \in \U} c\, u. 
$$
When used with a matrix argument, $h_{\U}(D)$ is understood as
$$
h_{\U}(D) := \begin{pmatrix}
h_{\U}(e_1^T D) & \dots & h_{\U}(e_q^T D)
\end{pmatrix}^T.
$$
Eventually, we use the following shorthand notations.
A convex and compact set containing the origin as an interior point is called a C-set. 
The matrix $I_p$ refers to the identity matrix in $\R^{p \times p}$. 

\section{Preliminaries}

In the following, we specify the definition of RPI sets from the introduction.

\begin{defn}
\label{def:RPI}
The set $\Pp$ is called robust positively invariant (RPI) for system~\eqref{eq:system} with constraints~\eqref{eq:constraints}, if (i) $\Pp \subseteq \X$ and (ii) $A\, x+w\in \Pp$ for every $x \in \Pp$ and every $w \in \W^\alpha$.
\end{defn}

Note that both the union and the intersection of two RPI sets result in another RPI set. Further note that $\Pp=\emptyset$ is a RPI set according to Def.~\ref{def:RPI}. Based on these observations, the following definitions of the maximal and minimal RPI set are reasonable. 

\begin{defn}
\label{def:MRPI}
The union of all RPI sets for~\eqref{eq:system} with~\eqref{eq:constraints} is called the maximal robust positively invariant (MRPI) set for~\eqref{eq:system} with~\eqref{eq:constraints} and denoted by $\Pp_{\max}^\alpha$.
\end{defn}

\begin{defn}
\label{def:mRPI}
The intersection of all nonempty RPI sets and the MRPI set for~\eqref{eq:system} with~\eqref{eq:constraints} is called the minimal robust positively invariant (mRPI) set for~\eqref{eq:system} with~\eqref{eq:constraints} and denoted by $\Pp_{\min}^\alpha$.
\end{defn}

Two important statements can be inferred  from Defs.~\ref{def:MRPI} and \ref{def:mRPI}. First, $\Pp_{\min}^\alpha$ is empty if and only if $\Pp_{\max}^\alpha$ is empty. Second, $\Pp_{\min}^\alpha \subseteq \Pp_{\max}^\alpha$.
In order to characterize $\Pp_{\max}^\alpha$ and $\Pp_{\min}^\alpha$ more precisely, we will analyze two special sequences of sets (inspired by the sequences in \cite[Eqs. (1.10) and (4.3)]{Kolmanovsky1998}).
The first sequence 
\begin{equation}
\label{eq:sequenceRk}
\Rr_{k+1}^\alpha := A\, \Rr_k^\alpha \oplus \W^\alpha \qquad \text{with} \qquad  \Rr_{0}^\alpha:=\{0\}.
\end{equation}
 provides sets $\Rr_k^\alpha \subseteq \R^n$ that contain states $x^\ast$ for which there exist disturbances $w(k) \in \W^\alpha$ such that
$x^\ast = \sum_{j=0}^{k-1} A^{k-1-j}\,w(j)$.
In other words, $\Rr_k^\alpha$ contains states that are reachable  from the origin within $k$ time steps.
The second sequence
\begin{equation}
\label{eq:sequenceSk}
\Ss_{k+1}^\alpha := A^{-1} (\Ss_k^\alpha \ominus \W^\alpha) \cap \X \quad \text{with} \quad  \Ss_{0}^\alpha:=\X,
\end{equation}
defines sets $\Ss_k^\alpha \subseteq \R^n$ that contain initial conditions $x(0)$ for which the trajectory of the disturbed system~\eqref{eq:system} remains in $\X$ for at least $k$ time steps and for all disturbances in $\W^{\alpha}$.
More precisely, all states in $\Ss_k^\alpha$ satisfy the following condition.

\begin{lem}
\label{lem:xInSkAlpha}
Let  $k \in \N_+$  and let $x_0 \in \X$. Then $x_0 \in \Ss_k^\alpha$ if and only if
$$
A^j x_0 + \sum_{i=0}^{j-1} A^{j-1-i} w(i)  \in \X
$$
for every $j \in \N_{[1,k]}$ and every disturbance sequence $w(0),\dots, w(k-1) \in \W^{\alpha}$.
\end{lem}

Now, we obviously have $\Rr_k^\alpha \subseteq \Rr_{k+1}^\alpha$.
and $\Ss_{k+1}^\alpha \subseteq \Ss_{k}^\alpha$ for every $k \in \N$.
Thus, according to~\cite[p. 21]{Hausdorff1957}, the limits of the sequences~\eqref{eq:sequenceRk} and~\eqref{eq:sequenceSk} evaluate to 
\begin{align}
\label{eq:RLimit}
\Rr_\infty^\alpha &:=\lim_{k \rightarrow \infty} \Rr_k^\alpha =  \bigcup_{k=0}^\infty \Rr_k^\alpha \quad \text{and} \\
\label{eq:SLimit}
\Ss_\infty^\alpha &:=\lim_{k \rightarrow \infty} \Ss_k^\alpha  = \bigcap_{k=0}^\infty \Ss_k^\alpha. 
\end{align}
It is well-known that these limits are closely related to $\Pp_{\max}^\alpha$ and $\Pp_{\min}^\alpha$. In fact, according to Lem.~\ref{lem:PmaxPmin}, $\Ss_\infty^\alpha$ equals the MRPI set and, if $\Rr_{\infty}^\alpha\subseteq \X$, $\Rr_\infty^\alpha$ describes the mRPI set.

\begin{lem}
\label{lem:PmaxPmin}
Let $\Rr_{\infty}^\alpha$ and $\Ss_\infty^\alpha$ be defined as in~\eqref{eq:RLimit} and~\eqref{eq:SLimit}. Then, $\Pp_{\max}^\alpha=\Ss_\infty^\alpha$ and 
\begin{equation}
\label{eq:Pmin}
\Pp_{\min}^\alpha=\left\{ \begin{array}{ll}
\Rr_{\infty}^\alpha & \text{if} \quad \Rr_{\infty}^\alpha\subseteq \X, \\
\emptyset & \text{otherwise}.
\end{array} \right.
\end{equation}
\end{lem}

While both sequences~\eqref{eq:sequenceSk} and~\eqref{eq:sequenceRk} are linked to RPI sets, it seems 
they are not closely related to one another.
This conclusion is wrong, however, since we have 
\begin{equation}
\label{eq:relationSkRk}
\Ss_{k}^\alpha = \bigcap_{j=0}^k (A^{j})^{-1} (\X \ominus \Rr_{j}^\alpha).
\end{equation}
according to \cite[Eqs.~(5.1) and (5.2)]{Kolmanovsky1998}.
Since our setup slightly differs from \cite{Kolmanovsky1998}, we formally prove relation~\eqref{eq:relationSkRk} in the appendix for completeness. Finally, having~\eqref{eq:relationSkRk}, it is easy to see that 
\begin{equation}
\label{eq:relationSkRkPlus1}
\Ss_{k+1}^\alpha = (A^{k+1})^{-1} (\X \ominus \Rr_{k+1}^\alpha) \cap \Ss_k^\alpha
\end{equation}
holds for every $k \in \N$.


%\begin{assum}
%\label{assum:AXW}
%The eigenvalues $\lambda \in \C$ of the matrix $A \in \R^{n \times n}$ are strictly stable (i.e., $| \lambda|<1$).
%The sets $\X$ and $\W^\ast$ are C-sets in $\R^n$.
%\end{assum}


\section{Parametric robust positively invariant sets}
\label{sec:parametricRPIsets}

As sketched in the introduction, we intend to simultaneously analyze a family of RPI sets for system~\eqref{eq:system}$-$\eqref{eq:constraints} with different scaling factors $\alpha$.
Similar to \cite{Schaich2015}, we will describe these parametric sets in a lifted space of dimension $n+1$, where the first $n$ dimensions are linked to the state space and where the ($n+1$)-th dimension refers to the parameter $\alpha$.
As a preparation, we introduce the sets 
\begin{align}
\label{YIk}
\Y^{\I}_k &:= \left\{ \begin{pmatrix}
x \\
\alpha 
\end{pmatrix} \in \R^{n+1} \,\Big|\, x \in \Rr_k^\alpha,\,\,  \alpha \in \I \right\} \quad \text{and} \\
\label{ZIk}
\Z^{\I}_k &:= \left\{ \begin{pmatrix}
x \\
\alpha 
\end{pmatrix} \in \R^{n+1} \,\Big|\, x \in \Ss_k^\alpha,\,\, \alpha \in \I  \right\} 
\end{align}
for some (potentially degenerated) interval $\I \subset \R_+$ (e.g., $\I=[\lb{\alpha},\ub{\alpha}]$ or $\I=\alpha$).
Clearly, $\Y^{\I}_k$ and $\Z^{\I}_k$ can be understood as a parametric variant of $\Rr_k^\alpha$ and $\Ss_k^\alpha$ in the lifted space $\R^{n+1}$.
Consequently, the limits 
\begin{align}
\label{eq:YLimit}
\Y_\infty^\I &:=\lim_{k \rightarrow \infty} \Y_k^\I =  \bigcup_{k=0}^\infty \Y_k^\I \quad \text{and} \\
\label{eq:ZLimit}
\Z_\infty^\I &:=\lim_{k \rightarrow \infty} \Z_k^\I  = \bigcap_{k=0}^\infty \Z_k^\I 
\end{align}
describe parametric versions of mRPI and MPRI sets.
In \cite[Lem.~5.2]{Schaich2015} it is shown that $\Z_\infty^{[\lb{\alpha},\ub{\alpha}]}$ is a polytope if $\X$ and $\W^\ast$ are polytopes.
Here we show that $\Z_\infty^{[\lb{\alpha},\ub{\alpha}]}$ is a convex, compact, and full-dimensional set in $\R^n$ if $\X$ and $\W^\ast$ are C-sets (as assumed). Moreover, we show that $\Y_\infty^{[\lb{\alpha},\ub{\alpha}]}$ can be approximated arbitrarily close by a convex, compact, and full-dimensional set in $\R^{n+1}$.
In addition, this outer approximation will be such that the contained sets for a fixed $\alpha \in [\lb{\alpha},\ub{\alpha}]$ are RPI sets for or~\eqref{eq:system} with~\eqref{eq:constraints}.
We begin our analysis with two linked statements about the sets $\Y^{[\lb{\alpha},\ub{\alpha}]}_k$.

\begin{lem}
\label{lem:YalphaConvY1}
Let $\lb{\alpha},\ub{\alpha}\in \R_+$ with $\lb{\alpha} < \ub{\alpha} $ and let $k \in \N_+$. Then,
$
\Y^{[\lb{\alpha},\ub{\alpha}]}_k = \conv \left\{ 
\lb{\alpha}\,\Y^{1}_k , 
\ub{\alpha}\,\Y^{1}_k 
\right\}.
$
\end{lem}

\begin{proof}
Consider any $\alpha \in \R_+$ and note that
\begin{equation}
\label{eq:RkAlphaRk1}
\Rr_k^\alpha= \bigoplus_{j=0}^{k-1} A^j \,\W^\alpha = \alpha \bigoplus_{j=0}^{k-1} A^j \,\W^\ast = \alpha \, \Rr_k^1.
\end{equation}
We consequently have $\Y^{\alpha}_k =\alpha\,\Y^{1}_k $, which immediately proves the claim.
\end{proof}

\begin{thm}
\label{thm:YkCCF}
Let $\lb{\alpha},\ub{\alpha}\in \R_+$ with $\lb{\alpha} < \ub{\alpha} $ and let $k \in \N_+$. Then, $\Y^{[\lb{\alpha},\ub{\alpha}]}_k$ is a compact, convex and full-dimensional set in $\R^{n+1}$.
\end{thm}





\begin{proof}
Obviously, since $\W^\ast$ is a C-set in $\R^n$, the sets $\Rr_k^{1}$ are C-sets in $\R^n$ for every $k \in \N_+$. This observation in combination with Lem.~\ref{lem:YalphaConvY1} proves the claim.
\end{proof}

Compared to the statements in Lem.~\ref{lem:YalphaConvY1} and Thm.~\ref{thm:YkCCF}, the analysis of the sets $\Z^{[\lb{\alpha},\ub{\alpha}]}_k$ is slightly more complicated.
The reason is that the sets $\Ss_k^\alpha$ may be empty for some $\alpha \in \R_+$ and $k \in \N_+$. In fact, it is easy to see that $\Ss_k^\alpha$ will be empty for some $k \in \N_+$ if the scaling $\alpha$ is chosen bigger than the critical scaling
\begin{equation}
\label{eq:criticalAlpha}
\alpha^\ast :=  \sup_{\alpha} \,\alpha \quad \text{s.t.} \quad \alpha \, \Rr^1_\infty \subseteq \X.
\end{equation}
To see this, note that~\eqref{eq:criticalAlpha} in combination with~\eqref{eq:RkAlphaRk1} implies $\Rr^\alpha_\infty \nsubseteq \X$ for any $\alpha>\alpha^\ast$. Consequently, $\Pp_{\min}^\alpha = \emptyset$ according to \eqref{eq:Pmin}, which immediately leads to $\Pp_{\max}^\alpha = \emptyset$. Due to $\Pp_{\max}^\alpha=\Ss_\infty^\alpha$ and~\eqref{eq:SLimit}, this can only be the case if $\Ss_k^\alpha = \emptyset$ for a finite $k \in \N_+$. As a consequence of this observation, the scalings $\alpha$ are restricted by $\alpha^\ast$ in Thms.~\ref{thm:ZkCCF}, \ref{thm:outerApproxYinfI}, and \ref{thm:ZfinitelyDetermined}.

\begin{thm}
\label{thm:ZkCCF}
Let $\lb{\alpha},\ub{\alpha}\in \R_+$ with $\lb{\alpha} < \ub{\alpha} \leq \alpha^\ast$ and let $k \in \N$. Then, $\Z^{[\lb{\alpha},\ub{\alpha}]}_k$ is a compact, convex and full-dimensional set in $\R^{n+1}$.
\end{thm}

\begin{proof}
The claim obviously holds for $k=0$ since $\Z^{[\lb{\alpha},\ub{\alpha}]}_0 = \X \times [\lb{\alpha},\ub{\alpha}]$ and since $\X$ is a C-set in $\R^n$ by assumption.
For $k \in \N_+$, first note that $\Ss_k^{\alpha}$ is a compact, convex and full-dimensional set in $\R^n$ for every $\alpha \in [\lb{\alpha},\ub{\alpha}]$ (due to $0<\lb{\alpha} < \ub{\alpha} \leq \alpha^\ast$). Since $[\lb{\alpha},\ub{\alpha}]$ is a compact and full-dimensional set in $\R$, it is clear that $\Z^{[\lb{\alpha},\ub{\alpha}]}_k$ is a compact and full-dimensional set in $\R^{n+1}$. It remains to show that $\Z^{[\lb{\alpha},\ub{\alpha}]}_k$ is convex. By definition, this is the case if
\begin{equation}
\label{eq:zHat}
\hat{z}:=\xi z_1+(1-\xi) z_2 \in \Z^{[\lb{\alpha},\ub{\alpha}]}_k.
\end{equation}
for every $z_1,z_2 \in \Z^{[\lb{\alpha},\ub{\alpha}]}_k$ and $\xi \in [0,1]$. Clearly $z_1$ and $z_2$ can be decomposed as
$$
z_1 = \begin{pmatrix}
x_1 \\
\alpha_1
\end{pmatrix},
\quad 
z_2 = \begin{pmatrix}
x_2 \\
\alpha_2
\end{pmatrix}
 \quad \text{and} \quad 
\hat{z} = \begin{pmatrix}
\hat{x} \\
\hat{\alpha}
\end{pmatrix}
$$
with $\alpha_1,\alpha_2,\hat{\alpha} \in \R_+$. We have $x_1 \in \Ss_k^{\alpha_1}$ and $x_2 \in \Ss_k^{\alpha_2}$ by construction.
Analogously, \eqref{eq:zHat} holds if (and only if) $\hat{x} \in \Ss_k^{\hat{\alpha}}$. According to Lem.~\ref{lem:xInSkAlpha}, $\hat{x} \in \Ss_k^{\hat{\alpha}}$ if and only if
$$
A^j \hat{x} + \sum_{i=0}^{j-1} A^{j-1-i} \hat{w}(i)  \in \X
$$
for every $j \in \N_{[1,k]}$ and every disturbance sequence $\hat{w}(0),\dots, \hat{w}(k-1) \in \W^{\hat{\alpha}}$. Consider any such sequence and define
\begin{equation}
\label{eq:w12}
w_1(i):= \frac{\alpha_1}{\hat{\alpha}} \,\hat{w}(i)
\quad \text{and} \quad 
w_2(i):= \frac{\alpha_2}{\hat{\alpha}} \,\hat{w}(i)
\end{equation}
for every $i \in \N_{[0,k-1]}$. By definition of $\W^\alpha$, it is easy to see that
$$
w_1(0),..., w_1(k-1) \in \W^{\alpha_1}
\,\, \text{and} \,\,
w_2(0),..., w_2(k-1) \in \W^{\alpha_2}.
$$
Due to $x_1 \in \Ss_k^{\alpha_1}$ and $x_2 \in \Ss_k^{\alpha_2}$, we thus have
\begin{align}
\label{eq:x1Sequence}
A^j \,x_1 + \sum_{i=0}^{j-1} A^{j-1-i} \,w_1(i)  &\in \X  \quad \text{and}\\
\label{eq:x2Sequence}
A^j \,x_2 + \sum_{i=0}^{j-1} A^{j-1-i} \,w_2(i)  &\in \X
\end{align}
for every $j \in \N_{[1,k]}$  according to Lem.~\ref{lem:xInSkAlpha}.
Now, \eqref{eq:zHat} obviously yields
\begin{align}
\label{eq:xHat}
\hat{x} &=  \xi \,x_1 + (1-\xi) \,x_2 \quad \text{and}\\
\label{eq:alphaHat}
\hat{\alpha} &= \xi \,\alpha_1 + (1-\xi) \,\alpha_2.
\end{align}
Moreover, we find 
\begin{equation}
\label{eq:xiW12}
\xi w_1(i) + (1-\xi) \,w_2(i) = \frac{\xi \alpha_1 + (1-\xi) \alpha_2}{\hat{\alpha}} \, \hat{w}(i) =\! \hat{w}(i)\!
\end{equation}
due to~\eqref{eq:w12} and \eqref{eq:alphaHat}, respectively.
Equations \eqref{eq:x1Sequence}, \eqref{eq:x1Sequence}, and \eqref{eq:xiW12} in combination with convexity of $\X$ finally prove the second relation in~\eqref{eq:zHat}.
%choose any $\alpha_1,\alpha_2 \in [\lb{\alpha},\ub{\alpha}]$ with $\alpha_1 \leq \alpha_2$, $x_1 \in \Ss_k^{\alpha_1}$, and $x_2 \in \Ss_k^{\alpha_2}$. Let $\xi \in [0,1]$ and consider $\hat{x} = \xi \,x_1 + (1-\xi) \, x_2$ and $\hat{\alpha}=\xi \alpha_1 + (1-\xi) \, \alpha_2$. Then $\hat{x} \in \Ss_k^{\hat{\alpha}}$ and consequently 
%$$
%\begin{pmatrix}
%\hat{x} \\
%\hat{\alpha}
%\end{pmatrix} \in \Z^{[\lb{\alpha},\ub{\alpha}]}_k.
%$$
%To see this, assume that $\hat{x} \notin \Ss_k^{\hat{\alpha}}$
%
%$\Ss_k^{\alpha_1} \supseteq \Ss_k^{\hat{\alpha}} \supseteq  \Ss_k^{\alpha_2}$. Thus $x_2 \in \Ss_k^{\hat{\alpha}}$.
%Having $\hat{x} \notin \Ss_k^{\hat{\alpha}}$ consequently implies $x_1  \notin \Ss_k^{\hat{\alpha}}$ since $\Ss_k^{\hat{\alpha}}$ is convex.
\end{proof}

In the following theorem, we address the accurate outer approximation of the set $\Y^{[\lb{\alpha},\ub{\alpha}]}_\infty$.
The result can be understood as a parametric extension of \cite[Thm.~1]{Rakovic2005}. As a preperation, we introduce the scaling matrix
\begin{equation}
\label{eq:scalingMatrixE}
 E^\epsilon := \begin{pmatrix}
 (1+\epsilon) I_n & 0 \\ 0 & 1
 \end{pmatrix} \in \R^{(n+1) \times (n+1)}
\end{equation} 
for some $\epsilon \in \R_+$ and the projection matrix
$$
P:=\begin{pmatrix}
  I_n & 0 
 \end{pmatrix} \in \R^{n \times (n+1)}.
 $$

\begin{thm}
\label{thm:outerApproxYinfI}
Let $\lb{\alpha},\ub{\alpha}\in \R_+$ with $\lb{\alpha} < \ub{\alpha}< \alpha^\ast$, let $\epsilon \in \R_+$, and
 assume $M \in \N$ is such that
 \begin{equation}
\label{eq:conditionYEpsApproximation}
A^{M} \W^\ast \subseteq \mu \,\W^\ast \quad \text{with} \quad \mu:=\frac{\epsilon}{1+\epsilon}.
\end{equation}
 holds. Then,
$\Y^{[\lb{\alpha},\ub{\alpha}]}_\infty \subseteq E^\epsilon\Y^{[\lb{\alpha},\ub{\alpha}]}_M$.
 Moreover, if
 \begin{equation}
\label{eq:boundOnAlphaUb}
\ub{\alpha} \leq \frac{\alpha^\ast}{1+\epsilon},
\end{equation}
 $P E^\epsilon \Y^{\alpha}_M$ is an RPI set for~\eqref{eq:system} with~\eqref{eq:constraints} for every $\alpha \in [\lb{\alpha},\ub{\alpha}]$. 
\end{thm}

\begin{proof}
According to \cite[Thm.~1]{Rakovic2005}, \eqref{eq:conditionYEpsApproximation} yields
\begin{equation}
\label{eq:RInfty1EpsApproximation}
\Rr_\infty^1 \subseteq \frac{1}{1-\mu} \Rr_M^1 = (1+\epsilon) \, \Rr_M^1.
\end{equation}
Following~\eqref{eq:RkAlphaRk1}, relation~\eqref{eq:RInfty1EpsApproximation} implies
$$
\Rr_\infty^\alpha \subseteq  (1+\epsilon) \, \Rr_M^\alpha \quad \text{for every} \quad \alpha \in [\lb{\alpha},\ub{\alpha}].
$$
Obviously, $\Rr_\infty^\alpha \subseteq  (1+\epsilon) \, \Rr_M^\alpha $ can equivalently be stated as $
\Rr_\infty^\alpha \subseteq  (1+\epsilon) \,I_p\, \Rr_M^\alpha$ 
which proves $\Y^{[\lb{\alpha},\ub{\alpha}]}_\infty \subseteq E^\epsilon\Y^{[\lb{\alpha},\ub{\alpha}]}_M$. To see that \eqref{eq:boundOnAlphaUb} implies that $P E^\epsilon \Y^{\alpha}_M$ is an RPI set  for every $\alpha \in [\lb{\alpha},\ub{\alpha}]$, first note that $P E^\epsilon \Y^{\alpha}_M = (1+\epsilon) \Rr_M^\alpha$. Clearly, $(1+\epsilon) \Rr_M^\alpha \subseteq \X$ due to~\eqref{eq:boundOnAlphaUb}. Moreover, according to the proof of \cite[Thm.~1]{Rakovic2005}, \eqref{eq:conditionYEpsApproximation} implies 
$$ (1+\epsilon) A \, \Rr_M^\alpha \oplus \W^\alpha \subseteq (1+\epsilon) \Rr_M^\alpha,
$$
which proves that $(1+\epsilon) \Rr_M^\alpha$ is an RPI set  for~\eqref{eq:system} with~\eqref{eq:constraints} according to Def.~\ref{def:RPI}.
\end{proof}

Note that Thm.~\ref{thm:outerApproxYinfI} provides two statements.
First, it presents a method to accurately approximate $\Y^{[\lb{\alpha},\ub{\alpha}]}_\infty $. This is useful since 
$\Y^{[\lb{\alpha},\ub{\alpha}]}_\infty $ is generally not finitely determined. In other words, we may have $\Y^{[\lb{\alpha},\ub{\alpha}]}_\infty \subset \Y^{[\lb{\alpha},\ub{\alpha}]}_k $ for every $k \in \N$. Second, provided that the maximal scaling $\ub{\alpha}$ satisfies~\eqref{eq:boundOnAlphaUb}, the outer approximation of $\Y^{[\lb{\alpha},\ub{\alpha}]}_\infty $ contains RPI sets. \textcolor{red}{Add one sentence why this is cool}.
In contrast to $\Y^{[\lb{\alpha},\ub{\alpha}]}_\infty $,  the 
set $\Z^{[\lb{\alpha},\ub{\alpha}]}_\infty $ is always finitely determined as summarized in the following theorem.


\begin{thm}
\label{thm:ZfinitelyDetermined}
Let $\lb{\alpha},\ub{\alpha}\in \R_+$ with $\lb{\alpha} < \ub{\alpha} < \alpha^\ast$ and assume $N \in \N$ is such that
 \begin{equation}
\label{eq:conditionZFinitelyDetermined}
A^{N+1} \X \subseteq \eta \,\X \quad \text{with} \quad \eta:=1-\frac{\ub{\alpha}}{\alpha^\ast}
\end{equation}
holds.
 Then,  $\Z^{[\lb{\alpha},\ub{\alpha}]}_\infty = \Z^{[\lb{\alpha},\ub{\alpha}]}_N $. Moreover,   $P \Z^{\alpha}_N$ is the MRPI set for~\eqref{eq:system} with~\eqref{eq:constraints} for every $\alpha \in [\lb{\alpha},\ub{\alpha}]$. 
\end{thm}



%\begin{thm}
%Let $\lb{\alpha},\ub{\alpha}\in \R_+$ with $\lb{\alpha} < \ub{\alpha} < \alpha^\ast$. Then, the limit $\Z^{[\lb{\alpha},\ub{\alpha}]}_\infty $ is finitely determined, i.e., there exists a finite $N \in \N$ such that $\Z^{[\lb{\alpha},\ub{\alpha}]}_\infty = \Z^{[\lb{\alpha},\ub{\alpha}]}_N $.
%\end{thm}

\begin{proof}
We have $\Z^{[\lb{\alpha},\ub{\alpha}]}_\infty = \Z^{[\lb{\alpha},\ub{\alpha}]}_N $ if (and only if) 
\begin{equation}
\label{eq:ZNPlus1EqualsZN}
\Z^{[\lb{\alpha},\ub{\alpha}]}_{N+1}= \Z^{[\lb{\alpha},\ub{\alpha}]}_N.
\end{equation}
Condition~\eqref{eq:ZNPlus1EqualsZN} holds 
 if (and only if) 
\begin{equation}
\label{eq:SNPlus1EqualsSN}
 \Ss_{N+1}^\alpha = \Ss_{N}^\alpha \quad \text{for every} \quad \alpha \in [\lb{\alpha},\ub{\alpha}].
\end{equation}
  Now, from relation~\eqref{eq:relationSkRkPlus1}, it is easy to see that~\eqref{eq:SNPlus1EqualsSN} holds if (and only if)
 \begin{equation}
\label{eq:exactConditionZFinitelyDetermined}
  A^{N+1} \Ss_N^\alpha \subseteq \X \ominus \Rr_{N+1}^\alpha \quad \text{for every} \quad \alpha \in [\lb{\alpha},\ub{\alpha}].
\end{equation}
We next show that~\eqref{eq:exactConditionZFinitelyDetermined} holds if 
 \begin{equation}
\label{eq:auxiliaryConditionZFinitelyDetermined}
A^{N+1} \X \subseteq \X \ominus \Rr^{\ub{\alpha}}_\infty.
\end{equation}
To see this, note that  $A^{N+1} \Ss_N^\alpha \subseteq  A^{N+1} \Ss_N^{\lb{\alpha}} \subseteq A^{N+1} \X$ and $ \X \ominus \Rr_{N+1}^\alpha \supseteq \X \ominus \Rr_{N+1}^{\ub{\alpha}} \supseteq \X \ominus \Rr_{\infty}^\alpha $ for every $ \alpha \in [\lb{\alpha},\ub{\alpha}]$.
The r.h.s.~in \eqref{eq:auxiliaryConditionZFinitelyDetermined} can be further underestimated. In fact, due to $0<\ub{\alpha} < \alpha^\ast$, we have 
$$
\X \ominus \Rr^{\ub{\alpha}}_\infty \supseteq \eta \,\X  \quad \text{with} \quad \eta:=1-\frac{\ub{\alpha}}{\alpha^\ast} \in (0,1)
$$
according to~\eqref{eq:criticalAlpha} and \cite[Rem.~2.1]{Kolmanovsky1998}. Thus, \eqref{eq:auxiliaryConditionZFinitelyDetermined} and consequently~\eqref{eq:ZNPlus1EqualsZN} holds, if~\eqref{eq:conditionZFinitelyDetermined} is fulfilled.
\end{proof}

%It is important to note that the set $\Gamma\,\Y^{[\lb{\alpha},\ub{\alpha}]}_M $ from Thm.~\ref{thm:outerApproxYinfI} is
%
%\begin{align}
%\label{eq:sequenceTk}
%\T_{k+1}^\alpha &:= A^{-1} (\T_k^\alpha \ominus \W^\alpha) \cap \X \quad  \text{with} \\  
%\T_{0}^\alpha &:=\X \cap (1+\epsilon) \,\Rr_M^\alpha ,
%\end{align}
%
%
%$$
%\M^{\I}_k := \left\{ \begin{pmatrix}
%x \\
%\alpha 
%\end{pmatrix} \in \R^{n+1} \,\Big|\, x \in \T_k^\alpha,\,\, \alpha \in \I  \right\} 
%$$


\begin{rem}
Obviously, $M \in \N$ and $N \in \N$ satisfying~\eqref{eq:conditionYEpsApproximation} and~\eqref{eq:conditionZFinitelyDetermined}, respectively, can always be found since both $\mu$ and $\eta$ lie in the interval $(0,1)$, since the eigenvalues of $A$ are strictly stable, and since $\X$ and $\W^\ast$ are C-sets. Assume, for example, that $\lb{r}_w,\ub{r}_w \in \R_+$ denote the radii of a ball that is contained in $\W^\ast$ and a ball that contains $\W^\ast$, respectively. Then, \eqref{eq:conditionYEpsApproximation} holds if 
\begin{equation}
\label{eq:MBoundApprox}
\|A^M\|_2 \leq \frac{\mu \,\lb{r}_w}{\ub{r}_w}. 
\end{equation}
\end{rem}






\subsection{Approximation of the critical scaling}

To apply the results  in Thms.~\ref{thm:ZkCCF}, \ref{thm:outerApproxYinfI}, and \ref{thm:ZfinitelyDetermined}, we require (an approximation of) the critical scaling $\alpha^\ast$ in~\eqref{eq:criticalAlpha}. In general, the exact computation of $\alpha^\ast$ is demanding since $\Rr^1_\infty$ may not be finitely determined and since  $\Rr^1_\infty$ may be open (see the example in Sect.~\ref{subsec:example1}).
However, according to the following theorem, an accurate approximation of $\alpha^\ast$ can be evaluated without computing $\Rr^1_\infty$.

\begin{thm}
\label{thm:approxCritScaling}
Let $\epsilon \in \R_+$ and
 assume $M \in \N$ is such that~\eqref{eq:conditionYEpsApproximation} holds and define
 \begin{equation}
\label{eq:criticalAlphaApprox}
\widehat{\alpha}^\ast :=  \max_{\alpha} \,\alpha \quad \text{s.t.} \quad \alpha \, (1+\epsilon)\, \Rr^1_M \subseteq \X.
\end{equation}
Then, $\widehat{\alpha}^\ast \leq \alpha^\ast \leq (1+\epsilon)\, \widehat{\alpha}^\ast$.
\end{thm}

\begin{proof}
Analogously to the proof of Thm.~\ref{thm:outerApproxYinfI}, satisfaction of \eqref{eq:conditionYEpsApproximation} yields
\eqref{eq:RInfty1EpsApproximation}. In combination with~\eqref{eq:criticalAlphaApprox}, we thus obtain $ \widehat{\alpha}^\ast\,\Rr_\infty^1 \subseteq  \widehat{\alpha}^\ast (1+\epsilon) \, \Rr_M^1 \subseteq \X$ which implies $\widehat{\alpha}^\ast \leq \alpha^\ast $ according to~\eqref{eq:criticalAlpha}. Similarly, we have $\alpha^\ast \,\Rr_M^1 \subseteq \alpha^\ast \,\Rr_\infty^1 \subseteq \X$ by construction and thus $\alpha^\ast \leq (1+\epsilon)\, \widehat{\alpha}^\ast$.
\end{proof}


\section{Computation and numerical examples}



In Sect.~\ref{sec:parametricRPIsets}, we discussed properties of parametric RPI sets and the approximation of the critical scaling $\alpha^\ast$ for general C-sets $\X$ and $\W^\ast$.
However, the actual computation of the sets $\Y^{[\lb{\alpha},\ub{\alpha}]}_k$ and $\Z^{[\lb{\alpha},\ub{\alpha}]}_k$ and the approximation of $\alpha^\ast$ is demanding for this general setup. 
We thus consider polytopic sets $\X$ and $\W^\ast$ to illustrate the  findings in Sect.~\ref{sec:parametricRPIsets}. 
More precisely, we assume $\X$ and $\W^\ast$ can be written as
\begin{align}
\label{eq:polytopeX}
\X &= \{ x \in \R^n \,|\, H_x  x \leq d_x \}  \qquad  \text{and}  \\
\label{eq:polytopeW}
\W^\ast &=  \{ w \in \R^n \,|\, H_w  w \leq d_w \}
\end{align}
with $H_x \in \R^{L_x \times n}$, $d_x \in \R^{L_x}$,  $H_w \in \R^{L_w \times n}$.
For polytopic $\X$ and $\W^\ast$, it is well-known that the sets $\Rr_k^\alpha$ and $\Ss_k^\alpha$ are  polytopes as well.
As for $\Rr_k^\alpha$, this observation immediately follows from~\eqref{eq:RkAlphaRk1}. Regarding $\Ss_k^\alpha$, 
we find
$$
\Ss_k^\alpha \!=\!\bigg\{ x\in \R^n \bigg| H_x A^j x \leq d_x - \sum_{i=0}^{j-1} \!h_{\W^\alpha}(H_x A^i), 
 \forall j  \in \N_{[0,k]} \!\bigg\}
$$
according to  \cite[p. 342]{Kolmanovsky1998}.
Thereby, the term $h_{\W^\alpha} (H_x A^i)$ can be computed by solving $L_x$ linear programs. 
Having polytopic sets $\Rr_k^\alpha$ and taking Lem.~\ref{lem:YalphaConvY1} into account, immediately implies that $
\Y^{[\lb{\alpha},\ub{\alpha}]}_k$ is a polytope.
To see that $
\Z^{[\lb{\alpha},\ub{\alpha}]}_k$ is polytopic as well, first note that
\begin{equation}
\label{eq:hWalpha}
h_{\W^\alpha}(c) = \sup_{w \in \W^\alpha}  c \,w = \alpha \sup_{w^\ast \in \W^\ast}  c \, w^\ast = \alpha \, h_{\W^\ast}(c)
\end{equation}
for any $c \in \R^{1 \times n}$.
Using~\eqref{eq:hWalpha} in the above expression for $\Ss_k^\alpha$ and substituting the result in~\eqref{ZIk} yields
\begin{align}
\nonumber
\!\!\!\!\Z_k^{[\lb{\alpha},\ub{\alpha}]}= \bigg \{ z \in \R^{n+1} \,\bigg|\, \begin{pmatrix} H_x A^j & \sum \limits_{i=0}^{j-1} h_{\W^\ast}(H_x A^i)  \end{pmatrix} z \leq d_x, \!\!\!\!\!\!\!\!\!\!\!\!\\
\label{eq:ZkPolytope}
\lb{\alpha} \leq z_{n+1} \leq \ub{\alpha},  \, \forall  j \in \N_{[0,k]} \bigg\} \quad \quad \!\!\!\!\!\!\!\!\!\!
\end{align}
after some basic manipulations.
Clearly, $\Z_k^{[\lb{\alpha},\ub{\alpha}]}$ as in~\eqref{eq:ZkPolytope} is a polytope. 
This observation is discussed in more detail in~\cite{Schaich2015}.

Polytopic sets $\X$ and $\W^\ast$ do not only simplify the computation of $\Y^{[\lb{\alpha},\ub{\alpha}]}_k$ and $
\Z^{[\lb{\alpha},\ub{\alpha}]}_k$.
They also allow to efficiently compute $M$, $N$, and $\widehat{\alpha}^\ast$ as in~\eqref{eq:conditionYEpsApproximation}, \eqref{eq:conditionZFinitelyDetermined}, and~\eqref{eq:criticalAlphaApprox}. 
In fact, as discussed in \cite[Eq.~(10)]{Rakovic2005},
\eqref{eq:conditionYEpsApproximation} holds if and only if 
\begin{equation}
\label{eq:MBoundPrecise}
 h_{\W^\ast}(H_w A^M) \leq \mu \,d_w.
\end{equation}
In general, condition~\eqref{eq:MBoundPrecise} outperforms the more conservative condition~\eqref{eq:MBoundApprox}.
Analogously, \eqref{eq:conditionZFinitelyDetermined} holds if and only if 
\begin{equation}
\label{eq:NBoundPrecise}
 h_{\X}(H_x A^{N+1}) \leq \eta \,d_x.
\end{equation}
Finally, similar to \cite[Eq.~(12)]{Rakovic2005}, the solution of the optimization~\eqref{eq:criticalAlphaApprox} evaluates to
\begin{equation}
\label{eq:computeAlphaCritApprox}
\widehat{\alpha}^\ast = \min_{i \in \N_{[1,L_x]}} \frac{e_i^T d_x}{(1+\epsilon)\,\sum_{k=0}^{M-1} h_{\W^\ast}(e_i^T H_x A^k) }.
\end{equation}
Note that the explicit computation of $\Rr_M^1$ is not required to solve~\eqref{eq:criticalAlphaApprox}.
In the following, we analyze two examples to illustrate the statements in Lem.~\ref{lem:YalphaConvY1} and Thms.~\ref{thm:outerApproxYinfI}, \ref{thm:ZfinitelyDetermined}, and \ref{thm:approxCritScaling}.

\subsection{Example 1}
\label{subsec:example1}

We first consider system~\eqref{eq:system} with 
$A=0.5$
and the constraints $\X=[-2,2]$ and $\W^\ast=[-1,1]$, which was also analyzed in \cite[Exmp.~6.10]{Blanchini2008} (without state constraints). 
In order to apply Thm.~\ref{thm:outerApproxYinfI} to compute an outer approximation of the parametric mRPI set, we first choose the smallest $M \in \N$ such that~\eqref{eq:conditionYEpsApproximation} holds for $\epsilon = 0.1$. For this example, we easily find
$$
A^{4} \W^\ast\! = \![-0.0625,0.0625] \!\subseteq \mu \,\W^\ast \! = \! [-0.\overline{09},0.\overline{09}] \!\subset \! A^{3} \W^\ast
$$
and consequently $M=4$. The same result can be obtained by computing the smallest $M$ that satisfies~\eqref{eq:MBoundPrecise}. In this context, note that $\X$ and $\W^\ast$ can be written as in~\eqref{eq:polytopeX}--\eqref{eq:polytopeW} with
$$
H_x = H_w = \begin{pmatrix}
\blind{+}1 \\
-1
\end{pmatrix}, \quad d_x = \begin{pmatrix}
2 \\
2
\end{pmatrix}, \quad \text{and}\quad d_w = \begin{pmatrix}
1 \\
1
\end{pmatrix}.
$$
We next compute an (accurate) approximation of the critical scaling $\alpha^\ast$ and find $\widehat{\alpha}^\ast = 0.9697$ according to  Eqs.~\eqref{eq:criticalAlphaApprox} and \eqref{eq:computeAlphaCritApprox}. Theorem~\ref{thm:approxCritScaling} additionally states that $\alpha^\ast \in [1,1+\epsilon ] \,\widehat{\alpha}^\ast = [0.9697,1.0667]$. This is obviously true since we have $\alpha^\ast = 1$ for this example. To see this, note that we obtain the sets
\begin{equation}
\label{eq:RkExample1}
\Rr_k^\alpha=[-\rho_k^\alpha,\rho_k^\alpha] \quad \text{with} \quad\rho_k^\alpha:=\alpha \left(2 - 0.5^{k-1} \right)
\end{equation}
according to~\eqref{eq:sequenceRk}. Clearly, $\lim_{k\rightarrow \infty} \rho_k^\alpha = 2\,\alpha$ but $\rho_k^\alpha < 2\,\alpha$ for every $k \in \N$. We thus find $\Rr_\infty^1=(-2,2)$ and $\alpha^\ast = 1$ according to~\eqref{eq:criticalAlpha}.
We now compute the parametric set $\Y_M^{[\lb{\alpha},\ub{\alpha}]}$ for the interval $[\lb{\alpha},\ub{\alpha}] = [0.2,0.8]$.
 Note that $\ub{\alpha}$ is such that condition~\eqref{eq:boundOnAlphaUb} holds since 
$$
\ub{\alpha} = 0.8 \leq \frac{\widehat{\alpha}^\ast}{1+\epsilon} =0.8815 \leq  \frac{\alpha^\ast}{1+\epsilon} = 0.\overline{90}.
$$
According to Lem.~\ref{lem:YalphaConvY1}, the set $\Y_M^{[\lb{\alpha},\ub{\alpha}]}$ is completely described by $\Y^1_M$ which, on its own, is given by $\Rr_M^1$. Evaluating~\eqref{eq:RkExample1} for $k=M=4$ yields $\Rr_M^1=[-1.875,1.875]$ (see the blue bar in Fig.~\ref{fig:first:example}).
We thus obtain
$$
\Y^{[\lb{\alpha},\ub{\alpha}]}_M = \conv \left\{ 
0.2\begin{pmatrix}
\pm 1.875 \\
1
\end{pmatrix},\,0.8\begin{pmatrix}
\pm 1.875 \\
1
\end{pmatrix}
\right\},
$$
which refers to the blue dash-dotted polytope in Fig.~\ref{fig:first:example}.
The outer approximation of $\Y^{[\lb{\alpha},\ub{\alpha}]}_M $ finally results from scaling $\Y^{[\lb{\alpha},\ub{\alpha}]}_M$ with the matrix $E^\epsilon$ from~\eqref{eq:scalingMatrixE}, which leads to
$$
E^\epsilon \,\Y^{[\lb{\alpha},\ub{\alpha}]}_M = \conv \left\{ 
\begin{pmatrix}
\pm 0.4125 \\
0.2
\end{pmatrix},\,\begin{pmatrix}
\pm 1.65 \\
0.8
\end{pmatrix}
\right\}.
$$
This parametric set refers to the gray set in Fig.~\ref{fig:first:example}. Since condition~\eqref{eq:boundOnAlphaUb}
is satisfied, Thm.~\ref{thm:outerApproxYinfI} states that $P E^\epsilon \Y^{\alpha}_M$ is an RPI set for~\eqref{eq:system} with~\eqref{eq:constraints} for every $\alpha \in [\lb{\alpha},\ub{\alpha}]$. Consider for example $\alpha = 0.5$ and the set 
$$
P E^\epsilon \Y^{\alpha}_M = (1+\epsilon) \Rr_M^{\alpha}=[-1.0313,1.0313],
$$
which is depicted in green in Fig.~\ref{fig:first:example}.
It is easy to see that $[-1.0313,1.0313]$ is indeed an RPI for $\alpha=0.5$ since
\begin{align} 
\nonumber 
A \, [-1.0313,1.0313] \oplus \W^\alpha &= [-1.0156,1.0156] \\
\nonumber
&\subseteq [-1.0313,1.0313].
\end{align}

We next compute the parametric MRPI set for the same interval $[\lb{\alpha},\ub{\alpha}] = [0.2,0.8]$. To this end, we first compute an $N \in \N$ such that~\eqref{eq:conditionZFinitelyDetermined} holds. At this point, note that $\alpha^\ast$ is generally not exactly available. However, choosing an $N$ such that
\begin{equation}
A^{N+1} \X \subseteq \widehat{\eta} \,\X \quad \text{with} \quad\widehat{\eta}:= 1 - \frac{\ub{\alpha}}{\widehat{\alpha}^\ast}
\end{equation}
holds obviously implies satisfaction of\eqref{eq:conditionZFinitelyDetermined} (due to $\widehat{\eta}\leq \eta$).
For this example, we find $\widehat{\eta}= 0.1750$ and $N=2$ and consequently $\Z_\infty^{[\lb{\alpha},\ub{\alpha}]}=\Z_N^{[\lb{\alpha},\ub{\alpha}]}$ according to Thm.~\ref{thm:ZfinitelyDetermined}. Now, due to the simple dynamics, we obtain $\Ss_k^\alpha=\X=\Pp_{\max}^\alpha$ for every $k \in \N$ and every $\alpha \in (0,\alpha^\ast]=(0,1]$. We thus find $\Z_N^{[\lb{\alpha},\ub{\alpha}]} = \X \times [\lb{\alpha},\ub{\alpha}]$ as illustrated in Fig.~\ref{fig:first:example}. 
Clearly, the additional statement in Thm.~\ref{thm:ZfinitelyDetermined} that 
 $P \Z^{\alpha}_N = \Pp_{\max}^\alpha$ for every $\alpha \in [\lb{\alpha},\ub{\alpha}]$, trivially holds for this simple example.

%For this simple example, which was also analyzed in \cite[Exmp.~6.10]{Blanchini2008} (without state constraints), the sets $\Rr_k^\alpha$ and $\Ss_k^\alpha$ can be explicitly stated. 
%In fact, it is easy to prove that we obtain 
%$$
%\Rr_k^\alpha=[-\rho_k^\alpha,\rho_k^\alpha] \quad \text{with} \quad\rho_k^\alpha:=\alpha \, (2 - 0.5^k) 
%$$
%and
%$$
%\Ss_k^\alpha = \left\{ \begin{array}{ll}
%\X & \text{if} \quad \alpha \leq 1, \\
%\,\![-\sigma_k^\alpha,\sigma_k^\alpha] & \text{if} \quad \alpha > 1 \,\, \text{and} \,\, k \leq \left\lfloor \log_2 \left( \frac{2\,\alpha}{\alpha-1} \right)\right\rfloor,\\
%\emptyset & \text{otherwise}
%\end{array}\right.
%$$
%with 
%$$
%\sigma_k^\alpha:=2\,\alpha -2^{k+1} (\alpha-1).
%$$
%In order to compute a




%$\widehat{\alpha}^\ast=...$ for $\epsilon=0.1$, $\mu=\frac{1}{11}$
%
%$\lb{\alpha}=0.2$,
%$\ub{\alpha}=0.8$
%
%Since
%$$
%\epsilon \leq  \frac{\widehat{\alpha}^\ast}{\ub{\alpha}}-1 = .. \leq \frac{\alpha^\ast}{\ub{\alpha}}-1 = 0.25,
%$$



\subsection{Example 2}

\textcolor{red}{ONLY NOTES IN THE FOLLOWING}


 \cite[Sect.~4.1]{Mayne2005} 

$\begin{pmatrix}
\blind{+}0.6696 & \blind{+}0.3369  \\ -0.6609 & -0.3261
\end{pmatrix}$  


  $\begin{array}{rl}
-4\!\!\! &\leq x_2 \leq 2 \\
-1\!\!\! &\leq K x \leq 1
\end{array}$  

 $\|w\|_\infty \leq 0.1$  
 
  $3.362391$    $3.362728$
  
 
 $\lb{\alpha}=0.5$
 
 $\ub{\alpha}=2.5$

\textcolor{red}{Merge second example section in previous one!}

%!TEX root = 2016_CDC.tex
\section{Numerical examples}

In this section we illustrate the obtained results with simple example systems.
%
First we consider the scalar system presented in~\cite[Exmp.~6.10]{Blanchini2008}:
%
\[
	x(k+1) = \frac{1}{2}x(k) + w
\]
%
where the state is constrained to the interval~$\X=[-2,2]$ and the disturbance has the nominal amplitude~$\W^\ast=[-1,1]$.
%
In this case the critical alpha can be explicitly calculated to be~$\alpha^\ast=1$ and both $\mathcal Z_\infty^{[0,\alpha^\ast]}$ and $\mathcal Y_\infty^{[0,\alpha^\ast]}$ can be explicitly determined.
%
The graph~$\mathcal Z^{[\frac{1}{2},1]}_\infty$ is illustrated in Figure~\ref{fig:first:example}.


The second system we consider is the two-dimensional example presented in~\cite[Sect.~4.1]{Mayne2005}:
%
\[
	x(k+1) = \left(\begin{array}{cc} 1 &1 \\ 0& 1\end{array}\right) x(k) + \left(\begin{array}{c}\frac{1}{2}\\1\end{array}\right) u(k) + w(k)
\]
%
where the input is given by the Riccati controller $u = Kx$  with $Q = I$ and $R=0.01$.
%
The constraints for this system are given by $\mathcal X = \{x\in\mathbb R^2\vert x_2\leq2,-1\leq Kx\leq 1\}$ and $\mathcal W^\ast = [-\frac{1}{10},\frac{1}{10}]\times [-\frac{1}{10},\frac{1}{10}]$.
%
For this setup the approximation of $\alpha^\ast$ leads to the lower and upper bounds $\underline\alpha = 3.362391$ and $\bar\alpha=3.362728$ respectively.
%
\begin{figure}
\centering
\begin{tikzpicture}
\draw[-latex'] (0,0) node[below] {$0$} -- (0,2) node[right] {$\alpha$};
\draw[-latex'] (-2.2,0) -- (2.2,0) node[right] {$x$};
\draw (.1,1) -- (-.1,1) node[above right] {$\alpha^\ast$};
\draw (-2,1) -- (0,0) -- (2,1);
\fill[pattern = north east lines, opacity=0.8] (-2,1) -- (0,0) -- (2,1) -- cycle;
\draw (2,.1) -- (2,-.1) node[below] {$2$};
\draw (-2,.1) -- (-2,-.1) node[below] {$-2$};
\end{tikzpicture}
\caption{$\mathcal Y_\infty^{[0,\alpha^\ast]}$ for $x(k+1) = \frac{1}{2}x(k) + w$.}\label{fig:first:example:mRPI}
\end{figure}
%
\begin{figure}\centering
\begin{tikzpicture}
	\draw[-latex'] (0,0) node[below] {$0$} -- (0,2) node[right] {$\alpha$};
	\draw[-latex'] (-2.2,0) -- (2.2,0) node[right] {$x$}; 
	\fill[pattern = north east lines, opacity=0.8] (-2,0) -- (2,0) node[below] {$2$} -- (2,1) -- (-2,1) -- cycle;
	\draw (2,0) -- (2,1);
	\draw (.1,1) -- (-.1,1) node[above right] {$\alpha^\ast$};
	\draw (-2,0) node[below] {$-2$} -- (-2,1);
\end{tikzpicture}
\caption{$\mathcal Z_\infty^{[0,\alpha^\ast]}$ for the $x(k+1) = \frac{1}{2}x(k) + w$.}\label{fig:first:example:MRPI}
\end{figure}

\begin{figure}
\tdplotsetmaincoords{80}{-30}
\centering
\begin{tikzpicture}[tdplot_main_coords]
\draw[-latex'] (0,0,0) -- (0,0,3.5) node[right] {$\alpha$};
\draw[-latex'] (0,0,0) -- (0,2.1,0) node[below] {$x_2$};
\draw[-latex'] (0,0,0) -- (3.5,0,0) node[below] {$x_1$};
\draw (2.5562,-0.5198,0) -- (-2.5000, 2.0000, 0) -- (-3.2716, 2.0000, 0) -- (-2.5562, 0.5198, 0) -- (3.5165, -2.5066,0) -- cycle;
\draw (-2.5000, 2.0000, 0) -- (-2.5000, 2.0000, 1.6849) -- (-3.2716, 2.0000, 0);
\draw (-2.5000, 2.0000, 1.6849) -- (-1.4878, 1.4956, 3.3627);
\draw (-2.5562, 0.5198, 0) -- (-0.5275, -0.4912, 3.3627);
\draw (3.5165, -2.5066,0) -- (1.4878, -1.4956, 3.3627);
\draw (2.5562,-0.5198,0) -- (0.5275,0.4912,3.3627);
\draw (1.4878, -1.4956, 3.3627) -- (0.5275,0.4912,3.3627) -- (-1.4878, 1.4956, 3.3627) -- (-0.5275, -0.4912, 3.3627) -- cycle;
\end{tikzpicture}
\caption{$\mathcal Z_\infty^{[0,\underline\alpha]}$}\label{fig:second:example}
\end{figure}

\begin{figure}\centering
\begin{tikzpicture}
\draw[-latex'] (0,0) -- (0,2) node[right] {$x_2$};
\draw[-latex'] (0,0) -- (2.5,0) node[below] {$x_1$};
\draw (1.9529,   -0.2191) -- (-2.5000,    2.0000) -- (-2.8137,    2.0000) -- (-1.9529,    0.2191) -- (2.9132,   -2.2059) -- cycle; % \alpha = 1
\draw (1.5397,   -0.0132) -- (-2.5000,    2.0000) -- (-2.5000,    2.0000) -- (-1.5397,    0.0132) -- (2.5000,   -2.0000) -- cycle; %\alpha = 1.6849
\draw[densely dotted] (2.0615,   -1.0928) -- (1.5397,   -0.0132) -- (-0.8800,    1.1927) -- (-2.2755,    1.8881) -- (-2.4459,    1.8881) -- (-1.5397,    0.0132) -- (1.1045,   -1.3046) -- (2.5000,   -2.0000) -- cycle; % interpolated
\draw (2.0082,   -1.7549) -- (1.0480,    0.2318) -- (-2.0082,    1.7549) -- (-1.0480,   -0.2318) -- cycle; %\alpha = 2.5
\end{tikzpicture}
\end{figure}

\section{Conclusion and outlook}
\label{sec:conclusionOutlook}






\bibliographystyle{ieeetr}
%\bibliographystyle{plain}        % Include this if you use bibtex 
\bibliography{msdLiterature} 

\appendix

\begin{lem}
\label{lem:SkRk}
Let $k \in \N$. Then, the set $\Ss_k^\alpha$ can be expressed as in~\eqref{eq:relationSkRk}.
\end{lem}

\begin{proof}
We prove the claim by induction. Relation~\eqref{eq:relationSkRk} obviously holds for $k = 0$ since we obtain $\Ss_0^\alpha = \X$ as in~\eqref{eq:sequenceSk}. It remains to show that~\eqref{eq:relationSkRk} implies
\begin{equation}
\label{eq:inductionStep}
\Ss_{k+1}^\alpha = \bigcap_{j=0}^{k+1} (A^{j})^{-1} (\X \ominus \Rr_{j}^\alpha).
\end{equation}
To this end, first note that~\eqref{eq:sequenceSk}  in combination with \eqref{eq:relationSkRk} yields
\begin{equation}
\label{eq:SkPlus1Rewritten}
\Ss_{k+1}^\alpha = A^{-1} \! \left( \left(\bigcap_{j=0}^k (A^{j})^{-1} (\X \ominus \Rr_{j}^\alpha) \right) \ominus \W^\alpha\right) \cap \X
\end{equation}
Now, the r.h.s.~in~\eqref{eq:SkPlus1Rewritten} can be rewritten as 
\begin{align}
\nonumber
&A^{-1} \!\left( \bigcap_{j=0}^k (A^{j})^{-1} (\X \ominus \Rr_{j}^\alpha \ominus A^j  \W^\alpha )  \right) \cap \X \\
\nonumber
&= \left( \bigcap_{j=0}^k (A^{j+1})^{-1} (\X \ominus \Rr_{j}^\alpha \ominus A^j  \W^\alpha )  \right) \cap \X \\
\nonumber
&=  \left( \bigcap_{j=1}^{k+1} (A^{j})^{-1} (\X \ominus \Rr_{j+1}^\alpha \ominus A^{j+1}  \W^\alpha )  \right) \cap \X 
\end{align}
Since $\Rr_{j+1}^\alpha \ominus A^{j+1}  \W^\alpha = \Rr_{j}^\alpha$ for every $j \in \N$ due to convexity and compactness of $\Rr_{j+1}^\alpha$, $A^{j+1}  \W^\alpha$, and $\Rr_{j}^\alpha$  (see \cite[p. 325]{Kolmanovsky1998}), we finally obtain
\begin{equation}
\label{eq:finalLine}
\Ss_{k+1}^\alpha = \left( \bigcap_{j=1}^{k+1} (A^{j})^{-1} (\X \ominus \Rr_{j}^\alpha )  \right) \cap \X 
\end{equation}
Clearly, \eqref{eq:inductionStep} and~\eqref{eq:finalLine} are equivalent.
\end{proof}



\end{document}
