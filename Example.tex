%!TEX root = 2016_CDC.tex
\section{Numerical examples}

In this section we illustrate the obtained results with simple example systems.
%
First we consider the scalar system presented in~\cite[Exmp.~6.10]{Blanchini2008}:
%
\[
	x(k+1) = \frac{1}{2}x(k) + w
\]
%
where the state is constrained to the interval~$\X=[-2,2]$ and the disturbance has the nominal amplitude~$\W^\ast=[-1,1]$.
%
In this case the critical alpha can be calculated to be~$\alpha^\ast=1$ and both $\mathcal Z_\infty^{[0,\alpha^\ast]}$ and $\mathcal Y_\infty^{[0,\alpha^\ast]}$ can be explicitly determined.
%
Although we can analytically calculate $\mathcal Y_\infty^{[0,1]}$ we illustrate the approximation using a finite iterate~$\mathcal Y_M^{[\underline\alpha,\bar\alpha]}$ in Figure~\ref{fig:first:example}, for comparison the set~$\mathcal Z_\infty^{[\underline\alpha,\bar\alpha]}$ is also depicted.


The second system we consider is the two-dimensional example presented in~\cite[Sect.~4.1]{Mayne2005}:
%
\[
	x(k+1) = \left(\begin{array}{cc} 1 &1 \\ 0& 1\end{array}\right) x(k) + \left(\begin{array}{c}\frac{1}{2}\\1\end{array}\right) u(k) + w(k)
\]
%
where the input is given by the Riccati controller $u = Kx$  with $Q = I$ and $R=0.01$.
%
The constraints for this system are given by $\mathcal X = \{x\in\mathbb R^2\vert x_2\leq2,-1\leq Kx\leq 1\}$ and $\mathcal W^\ast = [-\frac{1}{10},\frac{1}{10}]\times [-\frac{1}{10},\frac{1}{10}]$.
%
For this setup the approximation of $\alpha^\ast$ leads to the lower bound $\hat\alpha^\ast = 3.362391$.
%
We illustrate the set $\mathcal Z_\infty^{[0,\hat\alpha^\ast]}$ in Figure~\ref{fig:second:example}, we can see that the set is polytopic, i.e. there exists $0\leq\underline\alpha<\alpha<\bar\alpha\leq\hat\alpha^\ast$ such that the interpolation $\lambda\mathcal S_\infty^{\underline\alpha}\oplus(1-\lambda)\mathcal S_\infty^{\bar\alpha}$ is contained in $\mathcal S_\infty^{\lambda\underline\alpha+(1-\lambda)\bar\alpha}$, this constellation is illustrated in Figure~\ref{fig:containment}.
%
This confirms that in general a statement similar to Lemma~\ref{lem:YalphaConvY1} can not hold for~$\mathcal Z_k^{[\underline\alpha,\bar\alpha]}$.
%
\begin{figure}\centering\subfloat[$\mathcal Y_4^{[0.2,0.8]}$]{%
\begin{tikzpicture}[scale=1.1]
\draw[-latex'] (0,0) node[below] {$0$} -- (0,2) node[right] {$\alpha$};
\draw[-latex'] (-2.2,0) -- (2.2,0) node[right] {$x$};
\draw (.1,1) -- (-.1,1) node[above right] {$1$};
\draw[thick] (-1.9375,1) -- (1.9375,1);
\draw[latex'-] (-1.2,1) to[out=135,in=0] (-2,1.5) node[left] {$\mathcal R_4^1$};
\draw[dashed] (-2,-.1) node[below] {$-2$} -- (-2,1.2);
\draw[dashed] (2,-.1) node[below] {$2$} -- (2,1.2);
\fill[gray!30] (.4263,.2) -- (1.705,.8) -- (-1.705,.8) -- (-.4263,.2) -- cycle;
\draw[dashed,blue] (-1.55,.8) -- (-.3875,.2) -- (.3875,.2) -- (1.55,.8) -- cycle;
\draw[dotted] (2,1) -- (0,0) -- (-2,1);
\draw (.4263,.2) -- (1.705,.8) -- (-1.705,.8) -- (-.4263,.2) -- cycle;
\draw[latex'-] (.5,.6) to[out=60,in=180] (1,1.5) node[right] {$(1+\epsilon)\mathcal Y_4^{[0.2,0.8]}$};
\draw[latex'-] (.2,.1) to[out=0,in=90] (.7,-.1) node[below] {$\alpha\mathcal R_\infty^1$};
\end{tikzpicture}%
}\\
\subfloat[$\mathcal Z_1^{[0.2,0.8]}=\mathcal Z_\infty^{[0.2,0.8]}$]{%
\begin{tikzpicture}[scale=1.1]
	\fill[gray!30] (-2,.2) -- (2,.2) -- (2,.8) -- (-2,.8) -- cycle;
	\draw[-latex'] (0,0) node[below] {$0$} -- (0,2) node[right] {$\alpha$};
	\draw[-latex'] (-2.2,0) -- (2.2,0) node[right] {$x$};
	\draw (-2,.2) -- (2,.2) -- (2,.8) -- (-2,.8) -- cycle;
	\draw (2,0) -- (2,1);
	\draw (.1,1) -- (-.1,1) node[above right] {$1$};
	\draw[dashed] (-2,-.1) node[below] {$-2$} -- (-2,1.2);
	\draw[dashed] (2,-.1) node[below] {$2$} -- (2,1.2);
\end{tikzpicture}%
}
\caption{Both RPI sets for example system A can be calculated explicitly for all $\alpha\leq\alpha^\ast$. We illustrate the approximation to $\mathcal Y_\infty^{[0.2,0.8]}$. The MRPI set $\mathcal Z^{[0.2,0.8]}_\infty$ is determined after one iteration and is given by the state constraints.}
\label{fig:first:example}
\end{figure}
%

\begin{figure}
\tdplotsetmaincoords{75}{-45}
\centering
\begin{tikzpicture}[tdplot_main_coords]
\draw[-latex'] (0,0,0) -- (0,0,3.5) node[right] {$\alpha$};
\draw[-latex'] (0,0,0) -- (0,2.1,0) node[below] {$x_2$};
\draw[-latex'] (0,0,0) -- (3.5,0,0) node[below] {$x_1$};
\draw (2.5562,-0.5198,0) -- (-2.5000, 2.0000, 0) -- (-3.2716, 2.0000, 0) -- (-2.5562, 0.5198, 0) -- (3.5165, -2.5066,0) -- cycle;
\draw (-2.5000, 2.0000, 0) -- (-2.5000, 2.0000, 1.6849) -- (-3.2716, 2.0000, 0);
\draw (-2.5000, 2.0000, 1.6849) -- (-1.4878, 1.4956, 3.3627);
\draw (-2.5562, 0.5198, 0) -- (-0.5275, -0.4912, 3.3627);
\draw (3.5165, -2.5066,0) -- (1.4878, -1.4956, 3.3627);
\draw (2.5562,-0.5198,0) -- (0.5275,0.4912,3.3627);
\draw (1.4878, -1.4956, 3.3627) -- (0.5275,0.4912,3.3627) -- (-1.4878, 1.4956, 3.3627) -- (-0.5275, -0.4912, 3.3627) -- cycle;
\end{tikzpicture}
\caption{The MRPI set family of example system B~$\mathcal Z_\infty^{[0,\hat\alpha^\ast]}$, calculated after 2 iterations.}\label{fig:second:example}
\end{figure}

\begin{figure}\centering
\begin{tikzpicture}
\draw[loosely dotted, step=0.5] (-3,-2.5) grid (3,2);
\foreach \x in {-3,-2,...,3} \draw (\x,-2.4) -- (\x,-2.6) node[below] {$\x$};
\foreach \y in {-2,-1,...,2} \draw (-2.9,\y) -- (-3.1,\y) node[left] {$\y$};
\draw[-latex'] (-3,-2.5) -- (-3,2.2) node[right] {$x_2$};
\draw[-latex'] (-3,-2.5) -- (3.4,-2.5) node[below] {$x_1$};
\draw[thick] (1.9529,   -0.2191) -- (-2.5000,    2.0000) -- (-2.8137,    2.0000) -- (-1.9529,    0.2191) -- (2.9132,   -2.2059) -- cycle; % \alpha = 1
\draw[thick] (1.5397,   -0.0132) -- (-2.5000,    2.0000) -- (-2.5000,    2.0000) -- (-1.5397,    0.0132) -- (2.5000,   -2.0000) -- cycle; %\alpha = 1.6849
\draw (2.0615,   -1.0928) -- (1.5397,   -0.0132) -- (-0.8800,    1.1927) -- (-2.2755,    1.8881) -- (-2.4459,    1.8881) -- (-1.5397,    0.0132) -- (1.1045,   -1.3046) -- (2.5000,   -2.0000) -- cycle; % interpolated
\draw[thick] (2.0082,   -1.7549) -- (1.0480,    0.2318) -- (-2.0082,    1.7549) -- (-1.0480,   -0.2318) -- cycle; %\alpha = 2.5
\draw[latex'-] (-2.6418, 1.6438) to[out=-100,in=180] (-1.5,-1) node[right] {$\mathcal S_\infty^{2.5}$};
\draw[latex'-] (1.8160, -1.3576) to[out=180,in=60] (1,-1.5) node[below] {$\mathcal S_\infty^1$};
\draw[latex'-] (-2.3605, 1.8880) to[out=-60,in=180] (0,1) node[right] {$\lambda\mathcal S_\infty^1\oplus(1-\lambda)\mathcal S_\infty^{2.5}$};
\draw[latex'-] (-2.3875, 1.9440) to[out=30,in=160] (0,1.5) node[right] {$\mathcal S_\infty^{1.6849}$};
\end{tikzpicture}
\caption{Example that a statement similar to Lemma~\ref{lem:YalphaConvY1} does not hold for $\mathcal Z_\infty^{[\underline\alpha,\bar\alpha]}$.
%
The set $\mathcal Z_\infty^{[\underline\alpha,\bar\alpha]}$ for a polytopic setup is itself a poyltope, therefore $0\leq\underline\alpha<\bar\alpha\leq\alpha^\ast$ can be found such that $\lambda\mathcal S_\infty^{\underline\alpha}\oplus(1-\lambda)\mathcal S_\infty^{\bar\alpha}\subseteq\mathcal S_\infty^{\lambda\underline\alpha+(1-\lambda)\bar\alpha}$.} 
\label{fig:containment}
\end{figure}